% Arquivo LaTeX de exemplo de dissertação/tese a ser apresentada à CPG do IME-USP
%
% Criação: Jesús P. Mena-Chalco
% Revisão: Fabio Kon e Paulo Feofiloff
% Adaptação para UTF8, biblatex e outras melhorias: Nelson Lago
%
% Except where otherwise indicated, these files are distributed under
% the MIT Licence. The example text, which includes the tutorial and
% examples as well as the explanatory comments in the source, are
% available under the Creative Commons Attribution International
% Licence, v4.0 (CC-BY 4.0) - https://creativecommons.org/licenses/by/4.0/


%%%%%%%%%%%%%%%%%%%%%%%%%%%%%%%%%%%%%%%%%%%%%%%%%%%%%%%%%%%%%%%%%%%%%%%%%%%%%%%%
%%%%%%%%%%%%%%%%%%%%%%%%%%%%%%% PREÂMBULO LaTeX %%%%%%%%%%%%%%%%%%%%%%%%%%%%%%%%
%%%%%%%%%%%%%%%%%%%%%%%%%%%%%%%%%%%%%%%%%%%%%%%%%%%%%%%%%%%%%%%%%%%%%%%%%%%%%%%%

% A opção twoside (frente-e-verso) significa que a aparência das páginas pares
% e ímpares pode ser diferente. Por exemplo, as margens podem ser diferentes ou
% os números de página podem aparecer à direita ou à esquerda alternadamente.
% Mas nada impede que você crie um documento "só frente" e, ao imprimir, faça
% a impressão frente-e-verso.
%
% Aqui também definimos a língua padrão do documento
% (a última da lista) e línguas adicionais.
%\documentclass[12pt,twoside,brazilian,english]{book}
\documentclass[12pt,oneside,english,brazilian]{book}

% Ao invés de definir o tamanho das margens, vamos definir os tamanhos do
% texto, do cabeçalho e do rodapé, e deixamos a package geometry calcular
% o tamanho das margens em função do tamanho do papel. Assim, obtemos o
% mesmo resultado impresso, mas com margens diferentes, se o tamanho do
% papel for diferente.
\usepackage[a4paper]{geometry}

\geometry{
  textwidth=152mm,
  hmarginratio=12:17, % 24:34 -> com papel A4, 24mm + 152mm + 34mm = 210mm
  textheight=237mm,
  vmarginratio=8:7, % 32:28 -> com papel A4, 32mm + 237mm + 28mm = 297mm
  headsep=11mm, % distância entre a base do cabeçalho e o texto
  headheight=21mm, % qualquer medida grande o suficiente, p.ex., top - headsep
  footskip=10mm,
  marginpar=20mm,
  marginparsep=5mm,
}

% Vários pacotes e opções de configuração genéricos; para personalizar o
% resultado, modifique estes arquivos.
\input{extras/basics}
\input{extras/languages}
\input{extras/feausp-thesis} % capa, páginas preliminares e alguns detalhes
\input{extras/floats}
\input{extras/fonts}
\input{extras/imeusp-formatting}
\input{extras/index}
\input{extras/bibconfig}
\input{extras/hyperlinks}
%\nocolorlinks % para impressão em P&B
%\input{extras/source-code}
\input{extras/utils}
\usepackage{accents}

% Diretórios onde estão as figuras; com isso, não é preciso colocar o caminho
% completo em \includegraphics (e nem a extensão).
\graphicspath{{figuras/},{logos/}}

% Comandos rápidos para mudar de língua:
% \en -> muda para o inglês
% \br -> muda para o português
% \texten{blah} -> o texto "blah" é em inglês
% \textbr{blah} -> o texto "blah" é em português
\babeltags{br = brazilian, en = english}

% Bibliografia
\usepackage[
  style=extras/plainnat-ime, % variante de autor-data, similar a plainnat
  %style=alphabetic, % similar a alpha
  %style=numeric, % comum em artigos
  %style=authoryear-comp, % autor-data "padrão" do biblatex
  %style=apa, % variante de autor-data, muito usado
  %style=abnt,
]{biblatex}


%%%%%%%%%%%%%%%%%%%%%%%%%%%%%%%%%%%%%%%%%%%%%%%%%%%%%%%%%%%%%%%%%%%%%%%%%%%%%%%%
%%%%%%%%%%%%%%%%%%%%%%%%%%%%%%%%%% METADADOS %%%%%%%%%%%%%%%%%%%%%%%%%%%%%%%%%%%
%%%%%%%%%%%%%%%%%%%%%%%%%%%%%%%%%%%%%%%%%%%%%%%%%%%%%%%%%%%%%%%%%%%%%%%%%%%%%%%%

% O arquivo com os dados bibliográficos para biblatex; você pode usar
% este comando mais de uma vez para acrescentar múltiplos arquivos
\addbibresource{mono.bib}

% Este comando permite acrescentar itens à lista de referências sem incluir
% uma referência de fato no texto (pode ser usado em qualquer lugar do texto)
%\nocite{bronevetsky02,schmidt03:MSc, FSF:GNU-GPL, CORBA:spec, MenaChalco08}
% Com este comando, todos os itens do arquivo .bib são incluídos na lista
% de referências
%\nocite{*}

% É possível definir como determinadas palavras podem (ou não) ser
% hifenizadas; no entanto, a hifenização automática geralmente funciona bem
\babelhyphenation{documentclass latexmk soft-ware clsguide} % todas as línguas
\babelhyphenation[brazilian]{Fu-la-no}
\babelhyphenation[english]{what-ever}

% Estes comandos definem o título e autoria do trabalho e devem sempre ser
% definidos, pois além de serem utilizados para criar a capa, também são
% armazenados nos metadados do PDF.
\title{
    % Obrigatório nas duas línguas
    titlept={Lei 14.133 e o Registro de Preços},
    titleen={},
    % Opcional, mas se houver deve existir nas duas línguas
    subtitlept={um olhar sob a Teoria dos Leilões},
    subtitleen={},
}

\author{William Alexandre Miura Gnann}

% Para TCCs, este comando define o supervisor
\orientador{Prof. Dr. Pedro Henrique Thibes Forquesato}

% A página de rosto da versão para depósito (ou seja, a versão final
% antes da defesa) deve ser diferente da página de rosto da versão
% definitiva (ou seja, a versão final após a incorporação das sugestões
% da banca).
\defesa{
  nivel=tcc, % mestrado, doutorado ou tcc
  programa={Ciências Econômicas},
  local={São Paulo},
  data=2024-06-06, % YYYY-MM-DD
  % A licença do seu trabalho. Use CC-BY, CC-BY-NC, CC-BY-ND, CC-BY-SA,
  % CC-BY-NC-SA ou CC-BY-NC-ND para escolher a licença Creative Commons
  % correspondente (o sistema insere automaticamente o texto da licença).
  % Se quiser estabelecer regras diferentes para o uso de seu trabalho,
  % converse com seu orientador e coloque o texto da licença aqui, mas
  % observe que apenas TCCs sob alguma licença Creative Commons serão
  % acrescentados ao BDTA. Se você tem alguma intenção de publicar o
  % trabalho comercialmente no futuro, sugerimos a licença CC-BY-NC-ND.
  direitos={CC-BY}, % Creative Commons Attribution 4.0 International License
}

\newtheorem{lema}{Lema}
\newtheorem{teorema}{Teorema}

%%%%%%%%%%%%%%%%%%%%%%%%%%%%%%%%%%%%%%%%%%%%%%%%%%%%%%%%%%%%%%%%%%%%%%%%%%%%%%%%
%%%%%%%%%%%%%%%%%%%%%%% AQUI COMEÇA O CONTEÚDO DE FATO %%%%%%%%%%%%%%%%%%%%%%%%%
%%%%%%%%%%%%%%%%%%%%%%%%%%%%%%%%%%%%%%%%%%%%%%%%%%%%%%%%%%%%%%%%%%%%%%%%%%%%%%%%

\begin{document}

%%%%%%%%%%%%%%%%%%%%%%%%%%% CAPA E PÁGINAS INICIAIS %%%%%%%%%%%%%%%%%%%%%%%%%%%%

% Aqui começa o conteúdo inicial que aparece antes do capítulo 1, ou seja,
% página de rosto, resumo, sumário etc. O comando frontmatter faz números
% de página aparecem em algarismos romanos ao invés de arábicos e
% desabilita a contagem de capítulos.
\frontmatter

\pagestyle{plain}

\onehalfspacing % Espaçamento 1,5 na capa e páginas iniciais

\maketitle % capa e folha de rosto

%%%%%%%%%%%%%%%% DEDICATÓRIA, AGRADECIMENTOS, RESUMO/ABSTRACT %%%%%%%%%%%%%%%%%%

%\begin{dedicatoria}
%Esta seção é opcional e fica numa página separada; ela pode ser usada para
%uma dedicatória ou epígrafe.
%\end{dedicatoria}

% Reinicia o contador de páginas (a próxima página recebe o número "i") para
% que a página da dedicatória não seja contada.
\pagenumbering{roman}

% As palavras-chave são obrigatórias, em português e em inglês, e devem ser
% definidas antes do resumo/abstract. Acrescente quantas forem necessárias.
\palavrachave{leilão}
\palavrachave{licitação}
\palavrachave{Sistema de Registro de Preços}

\keyword{auction}
\keyword{procurement}
\keyword{Sistema de Registro de Preços}

% O resumo é obrigatório, em português e inglês. Estes comandos também
% geram automaticamente a referência para o próprio documento, conforme
% as normas sugeridas da USP.
\resumo{
Este trabalho estuda a Lei de Licitações e especialmente o Sistema de Registro de Preços sob a ótica da Teoria dos Leilões. Desenvolve-se duas extensões para os pregões: uma onde o órgão público pode não comprar o bem contratado considerando o seu preço depois do certame e outra que leva em consideração a qualidade nesse mesmo arcabouço de opção de compra. Na primeira situação, os lances são mais agressivos que se comparados à situação regular. Na segunda, a qualidade ofertada tende a ser maior que a qualidade mínima.
}

\abstract{
This work studies the brazilian's procurement law and the Sistema de Registro de Preços, a contract similar to a purchase option, using Auction Theory framework. Two extensions to the procurement process are developed: one where the public body has the option to not purchase the good considering its price after the procurement and other where the quality problem is addressed using the same framework. In the first situation, the bids are more aggressive than in the regular situation. In the second, the quality offered tends to be higher than the minimum allowed quality.
}


%%%%%%%%%%%%%%%%%%%%%%%%%%% LISTAS DE FIGURAS ETC. %%%%%%%%%%%%%%%%%%%%%%%%%%%%%

% Como as listas que se seguem podem não incluir uma quebra de página
% obrigatória, inserimos uma quebra manualmente aqui.
\makeatletter
\if@openright\cleardoublepage\else\clearpage\fi
\makeatother

% Todas as listas são opcionais; Usando "\chapter*" elas não são incluídas
% no sumário. As listas geradas automaticamente também não são incluídas por
% conta das opções "notlot" e "notlof" que usamos para a package tocbibind.

% Normalmente, "\chapter*" faz o novo capítulo iniciar em uma nova página, e as
% listas geradas automaticamente também por padrão ficam em páginas separadas.
% Como cada uma destas listas é muito curta, não faz muito sentido fazer isso
% aqui, então usamos este comando para desabilitar essas quebras de página.
% Se você preferir, comente as linhas com esse comando e des-comente as linhas
% sem ele para criar as listas em páginas separadas. Observe que você também
% pode inserir quebras de página manualmente (com \clearpage, veja o exemplo
% mais abaixo).
\newcommand\disablenewpage[1]{{\let\clearpage\par\let\cleardoublepage\par #1}}

% Nestas listas, é melhor usar "raggedbottom" (veja basics.tex). Colocamos
% a opção correspondente e as listas dentro de um grupo para ativar
% raggedbottom apenas temporariamente.
\bgroup
\raggedbottom

%%%%% Listas criadas automaticamente

% Você pode escolher se quer ou não permitir a quebra de página
%\listoffigures
\disablenewpage{\listoffigures}

% Você pode escolher se quer ou não permitir a quebra de página
%\listoftables
\disablenewpage{\listoftables}

% Esta lista é criada "automaticamente" pela package float quando
% definimos o novo tipo de float "program" (em utils.tex)
% Você pode escolher se quer ou não permitir a quebra de página
%\listof{program}{\programlistname}
%\disablenewpage{\listof{program}{\programlistname}}

% Sumário (obrigatório)
\tableofcontents

\egroup % Final de "raggedbottom"

%%%%%%%%%%%%%%%%%%%%%%%%%%%%%%%% CAPÍTULOS %%%%%%%%%%%%%%%%%%%%%%%%%%%%%%%%%%%%%

% Aqui vai o conteúdo principal do trabalho, ou seja, os capítulos que compõem
% a dissertação/tese. O comando mainmatter reinicia a contagem de páginas,
% modifica a numeração para números arábicos e ativa a contagem de capítulos.
\mainmatter

\pagestyle{mainmatter}

% Espaçamento simples
\singlespacing

\chapter{Revisão bibliográfica}
\label{cap:revisao}

\section{Leilões}

\emph{Dou-lhe uma! Dou-lhe duas! Dou-lhe três!} O leilão certamente figura entre as instituições mais antigas e mais conhecidas da história da humanidade. Situa-se, na Idade Antiga, desde 500a.C., na Babilônia, onde as famílias leiloavam as filhas para casamento, passando pelo leilão do Império Romano em 193 a.C., vencido por Didius Julianus\footnote{Talvez não tenha sido um bom investimento, pois Didius Julianus morreu dois meses depois, decapitado pelas legiões lideradas por Septimius Severus. Segundo \citet{krishna}, trata-se de um caso extremo da maldição do vencedor.} ao dar a maior doação \citet{Cassady2021-ac}.

Também é bastante presente na cultura popular\footnote{\url{https://tvtropes.org/pmwiki/pmwiki.php/Main/Auction}.} fazendo parte de séries de TV como \emph{Star Trek}, jogos de videogame como \emph{Diablo} e \emph{Final Fantasy}, animações como \emph{WiFI Ralph} e \emph{Futurama} e músicas como \emph{Leilão}\footnote{Glória Groove. Clipe em: \url{https://www.youtube.co/watch?v=X7cz9v6mHzg}.}. Não é um exagero dizer que o senso comum entende o conceito de leilão e que esse conceito tem mais de 2000 anos.

Hoje em dia os leilões são usados para diversas finalidades: venda de lotes de itens da Receita Federal, venda de ativos financeiros, venda de propaganda em sites da Internet\footnote{A Google utiliza: \url{https://support.google.com/google-ads/answer/142918}.} e a venda de bens em geral. O leilão é um método universal de venda, pois qual quer coisa possível de ser vendida pode ser vendida num leilão \citet{krishna}.

Duas propriedades importantes dos leilões são o fato extraírem informação dos agentes participantes por meio dos lances e a identidade desses agentes ser irrelevante para o resultado \citet{krishna}.

Os leilões entregam um arcabouço teórico geral para entender o problema da alocação de recursos entre agentes auto-interessados, tendo aplicações muito além das vendas como a alocação de recursos computacionais em sistemas compartilhados \citet{Shoham2008}.

\section{Leilões de um único objeto}

Trataremos aqui dos formatos de leilões mais conhecidos na literatura.

\subsection{Leilão inglês}

Provavelmente o formato de leilão mais conhecido é o leilão aberto ascendente ou também conhecido por leilão inglês. É o modelo onde há um leiloeiro e diversos compradores que dão lances cada vez maiores, cobrindo o lance anterior, até que ninguém queira cobrir o último lance. Há um tempo para que os interessados possam cobrir o lance anterior. O lance de maior valor, e portanto o último, determinará quem é o vencedor.

Uma variação do leilão inglês é o leilão japonês, onde o leiloeiro anuncia os preços e, para cada preço, os compradores decidem se participam até que somente um anuncie a participação para um dado preço \citet{Shoham2008}.

Com alguma reflexão podemos ver que o lance que vence o leilão precisa ser minimamente maior ou igual ao lance do segundo colocado.

\subsection{Leilão holandês}

Ao contrário do leilão inglês, os preços, no holandês, são diminuídos com o tempo. O leiloeiro começa anunciando um preço presumidamente alto e segue diminuindo até que alguém aceite participar. O leilão é chamado de holandês, pois é o procedimento utilizado no mercado de flores de Amsterdã e quem diminui o preço é um relógio que é parado quando alguém decide comprar as flores pelo preço exibido \citet{Shoham2008}.

\subsection{Leilão de envelope fechado}

O leilão de envelope fechado funciona exatamente como o nome diz: todos os participantes submetem envelopes com seus lances que são abertos ``ao mesmo tempo'' de tal sorte que nenhum participante observa o lance do outro até a abertura dos envelopes. Como há apenas um lance por participante, o tempo despendido para comunicação é bem menor \citet{karlin2017game}. A própria entrega de lances não precisa ser realizada de forma síncrona \citet{krishna}.

Os dois modelos de leilão de envelope fechado mais conhecidos são o de primeiro (ou maior) preço e o de segundo preço, onde o valor pago pelo participante vencedor corresponde ao lance do segundo colocado.

O leilão holandês e o de primeiro são estrategicamente equivalentes, isto é, possuem a mesma forma normal \citet{krishna}. Vale ressaltar que em ambos os formatos \emph{nenhuma} informação é dada até o momento em que há a determinação do vencedor.

Já os leilões inglês e de segundo preço são \emph{parecidos}, mas não equivalentes. Sob a hipótese de independência entre os valores privados, a estratégia ótima é a mesma \citet{krishna}. Contudo, pode haver alguma diferença considerando, por exemplo, o tamanho do passo no leilão inglês \citet{Shoham2008}.

\subsection{Valoração}

Se o leiloeiro soubesse exatamente quem atribui o maior valor ao produto que está a venda, não haveria sentido em se realizar um leilão, pois bastaria o leiloeiro vender o produto diretamente. Essa incerteza em relação à valoração é uma característica inerente ao leilão \citet{krishna}.

Dependendo de como é a configuração dos lances, os demais participantes passam a conhecer, mesmo que parcialmente, a valoração de seus adversários. Num leilão inglês, à medida em que os lances são dados, mais informação sobre a valoração fica disponível para todos os participantes. Por outro lado, num leilão de envelope fechado a valoração só é conhecida quando da abertura dos envelopes.

Esse acesso à informação pode mudar o comportamento dos agentes, pois pode revelar alguma informação sobre o bem à venda que não é conhecida para todos os participantes. O impacto depende da relação entre os agentes e o bem. É razoável que se o bem for para consumo, conhecer a valoração de outros agentes não é tão relevante. Entretanto, se o objetivo for revender, vale a pena conhecer a valoração dos outros participantes para se ter uma ideia do mercado. Essa informação também é importante na ocasião de o próprio agente ter uma mera crença sobre o valor do bem que está em disputa. À medida em que mais lances são conhecidos essa crença pode ser atualizada.

O caso onde a valoração dos demais participantes é irrelevante é chamado de \textbf{valores privados independentes}. Quando há alguma influência, chamamdos de \textbf{valores privados interdependentes}. Quando todos têm uma mesma valoração a situação é chamada de \textbf{valor comum} \citet{krishna}.

\section{O modelo de um leilão}

Dados $N$ jogadores, cada jogador $i$ atribui um valor privado $x_i$ ao objeto que está à venda e, portanto, não pagará mais do que $x_i$ por ele. Se o preço que venceu o leilão for $p$, o jogador que vencer ganhará $x_i - p$. Os outros jogadores ganharão zero. Não há restrição orçamentária sobre os agentes.

Primeiramente, vamos estudar como os jogadores determinam a função de lance, $\beta_i$. A função de lance depende exclusivamente do valor privado do jogador.

\subsection{O lance no leilão segundo preço}

\begin{lema}
    \label{estrategia-dominante}
    Num leilão de segundo preço, o jogador $i$ não tem incentivo para dar um lance diferente de sua valoração $x_i$. Isto é, $\beta_{i}(x_i) = x_i$.
\end{lema}

Em um leilão inglês, de uma forma parecida, só faz sentido o jogador $i$ permanecer dando lances enquanto o preço, $p$, for menor que $x_i$, pois isso maximiza sua utilidade independente do que fizerem os outros jogadores \citet{karlin2017game}. A estratégia do lema \ref{estrategia-dominante} também majora a utilidade do jogador $i$ independente dos outros jogadores. Em ambos os casos, a estratégia é uma estratégia dominante. É importante ressaltar que essa escolha acontece \emph{independentemente do perfil de risco} dos jogadores \citet{Shoham2008}.

\subsection{O lance no leilão de primeiro preço}

Como os leilões holandês e de primeiro preço são estrategicamente equivalentes, basta tratarmos o leilão de primeiro preço. Mas, ao contrário do leilão de segundo preço, não há uma estratégia dominante. Teremos de adicionar algumas propriedades ao modelo.

O jogador $i$ ainda conhece sua valoração $x_i$, mas, agora, também sabe que a valoração de todos os jogadores segue uma distribuição de probabilidade acumulada $F$, com derivada contínua e que assume valores não nulos em algum intervalo $[0, 1]$. Essa configuração confere o caráter bayesiano ao jogo, pois plugamos o conjunto de tipos e a distribuição de probabilidade à priori ao modelo. Além disso, vamos assumir que os jogadores são \textbf{neutros ao risco}.

Recapitulando, valem as propriedades \citet{krishna}:
\begin{itemize}
	\item Independência dos valores privados;
	\item Ausência de restrição orçamentária;
	\item Simetria entre jogadores;
	\item Neutralidade a risco.
\end{itemize}

Nosso objetivo é encontrar funções $\beta_i : [0, 1] \xrightarrow{} \mathbb{R}$ que representem em equilíbrio os lances dos jogadores $i$ definidas suas valorações. Desejamos encontrar o equilíbrio de Nash simétrico, isto é, todos os jogadores usarão a mesma função de lance, $\beta_i = \beta, \forall i$ \citet{krishna}.

Primeiramente, $\beta(0) = 0$, afinal nenhum jogador tem interesse em pagar por algo que vale 0 para si. Além disso, parece razoável supor que $\beta$ é crescente, pois, quanto mais valorizado o objeto, maior tende a ser o lance.

Seja $b$ o lance do jogador $i$. Se o jogador $i$ vencer, $b > max_{j \ne i} \beta(X_j)$. Seja $Y$ a variável aleatória associada à maior valoração dentre os jogadores restantes, então:
\begin{equation}
b > \beta(Y) \Rightarrow Y < \beta^{-1}(b)    
\end{equation}
Como todas $X_i$ são independentes, a distribuição de $Y$ é a estatística de ordem do máximo entre os $n-1$ outros jogadores, isto é, $G(y) = F(x) \ldots F(x) = F(x)^{n-1}$.

O \emph{payoff} esperado do jogador $i$ é:
\begin{equation}
    P[vencer] (x_i - b) \Rightarrow \
    G(\beta^{-1}(b)) (x_i - b)
\end{equation}

Basta encontrar o $b$ que maximiza o \emph{payoff}.

\begin{lema}
    A estratégia de equilíbrio simétrico num leilão de primeiro preço é dada por:
    \begin{equation}
        \beta(x) = \frac{1}{G(x)} \int_{0}^{x} yg(y)dy = E\left[ Y \mid Y<x \right] = x - \int_{0}^{x} \frac{G(y)}{G(x)}dy
    \end{equation}
\end{lema}

Vale ressaltar que o lance no leilão de primeiro preço é a valoração do jogador subtraída de um valor que é sempre menor do que $1$.

{ \huge por um gráfico }

Findos os lances, agora vamos determinar quanto o vencedor pagará em cada situação.

\subsection{O pagamento}

No leilão de primeiro preço é imediato, o pagamento esperado é dado pelo maior lance esperado, que é: $G(x)E\left[ Y \mid Y<x \right]$, onde $G(x)$ é a probabilidade de o jogador ganhar o certame com o lance $x$.

O pagamento no leilão de segundo preço é dado pelo segundo maior lance. Como $\beta(x) = x$, o pagamento esperado no leilão de segundo preço será dado pelo segundo maior valor, resultando em: $G(x)E\left[ Y \mid Y<x \right]$.

\begin{table}[]
	\centering
	\begin{tabular}{@{}llll@{}}
		\toprule
		tipo      & primeiro 					     & segundo      					\\ \midrule
		lance     & $E\left[ Y \mid Y<x \right]$     & $x$             					\\
		pagamento & $G(x)E\left[ Y \mid Y<x \right]$ & $G(x)E\left[ Y \mid Y<x \right]$ \\
		receita   & $E\left[ Y_2 \right]$            & $E\left[ Y_2 \right]$            \\
	\end{tabular}
	\caption{Comparativo entre os dois modelos de leilão.}
	\label{tab:tabela1}
\end{table}

\section{Leilões multi-objeto}

A principal simplificação do leilão unitário em relação ao leilão multi-objeto reside justamente no fato de não haver bens complementares nem bens substitutos nos lotes, uma vez que o objeto é unitário. Uma possibilidade de dar tratamento aos leilões multiunitários é justamente o leilão combinatório \citet{Shoham2008}. Contudo, há problemas práticos em se utilizar leilões combinatórios, pois sua solução exata é intratável de um ponto de vista computacional \citet{Nisan2007}. Aspectos jurídicos do emprego de licitações combinatórias, bem como um exemplo de seu uso no Brasil, são abordados em \citet{pellegrini2018:MSc}.

\section{Pregões}

Existem diversas formas de se abordar a otimização de processos licitatórios. Mas, talvez, a primeira pergunta a ser respondida, que aparece indiretamente em diversos artigos, é: qual a relação entre um leilão e um pregão? Tal pergunta foi formalizada em \citet{deCastro2010} com a prova de que a relação entre um pregão e um leilão é de dualidade, isto é, há um isomorfismo entre as estratégias empregadas num leilão (problema primal) e num pregão (problema dual). Esse resultado permite o emprego de resultados já consolidados na Teoria dos Leilões para atacar os problemas concernentes ao processo licitatório semelhantes aos da modalidade pregão.

Não parece haver uma extensa literatura sobre o sistema de registro de preços. Um exemplo é \citet{barbosa2013}. Barbosa analisa o sistema de registro de preços do ponto de vista dos licitantes e mostra que dependendo do custo, o carona pode inviabilizar o processo de compra como um todo. Também discute que o registro de preços é um mecanismo linear, diferentemente dos mecanismos usuais: uniforme, discriminatório e de \emph{Vickrey}. Algumas questões que não são contempladas versam justamente sobre quando o registro de preços é mais adequado que o pregão comum bem como problemas de leilões como a \emph{maldição do vencedor}: o custo efetivo do fornecimento dos bens no registro de preço acaba sendo maior que o preço negociado.

O problema da maldição do vencedor no contexto da nova lei de licitações é abordado por \citet{Signor2022}. A dinâmica de preços de reserva formado pela mediana de aquisições anteriores leva o preço a convergir ao preço mínimo exequível, ampliando o risco de o contrato não se concretizar. 

Uma discussão sobre o efeito da publicação do preço de reserva em relação à estratégia dos licitantes pode ser encontrada em \citet{Bugarin2022}. Em particular, a divulgação, ou não, do preço de reserva só se torna relevante quando se sabe o número de participantes no certame.

{\huge REESCREVER}

O efeito da negociação nos certames também é abordado. Entretanto, o artigo não aborda casos de licitações com mais de um item, bem como a aquisição de bens via registro de preços.

Há a preocupação com situações onde a qualidade do bem é relevante \citet{villa:2022}.

{\huge FALAR SOBRE o CHE, MANELLI E MCMILLAN}

A qualidade pode se tratar tanto da própria entrega do bem licitado quanto sobre alguma caraterística não-contratável, isto é, que não pode figurar diretamente no edital do certame, incentivando ofertantes de bens de baixa qualidade a atuarem de forma agressiva na fase de lances do certame. O primeiro caso pode configurar um problema de risco moral, pois pode ser mais vantajoso para o vencedor meramente não realizar a entrega do bem contratado. No contexto do registro de preços, pode acontecer justamente uma situação de maldição do vencedor, onde o fornecedor não consegue entregar todas as unidades do bem previstas no contrato. O segundo caso, por sua vez, é um caso de seleção adversa, uma vez que os licitantes com bens de maior qualidade não terão incentivos para participar do certame. Os autores propõem um mecanismo onde há a previsão não só de um preço de reserva, mas de um preço mínimo para a aquisição dos bens, o \emph{LoLA - Lowball Lottery Auction}.

\chapter{A Lei n\textordmasculine 14.133}
\label{cap:14133}

As compras no setor público são, em geral, disciplinadas pelo Artigo 37 da Constituição Federal, que diz, em seu inciso XXI, que \emph{as obras, serviços, compras e alienações serão contratados mediante processo de licitação pública}.

Desde o dia 1{\textordmasculine} de abril de 2021, a lei que estabelece as normas para os processos licitatórios é a Lei n{\textordmasculine} 14.133, que agregou o conteúdo das leis n{\textordmasculine} 8.666 de 1993, lei anterior de licitações, n{\textordmasculine} 10.520 de 2002, lei que disciplinava os procedimentos para pregões e n{\textordmasculine} 12.462 de 2011, que disciplinava os procedimentos para a contratação de grandes obras principalmente no contexto das obras associadas à Copa do Mundo de 2014. No Estado de São Paulo, a lei 14.133 passou a vigorar desde o dia 1{\textordmasculine} de janeiro de 2024.

Além de agregar diversas leis que tratam de logística pública, a nova lei conta com diversas inovações. A primeira delas é o caráter completamente digital: todos os documentos que compõem a instrução processual de uma contratação tramitam nos Sistemas de Compras do Governo Federal\footnote{\url{https://www.gov.br/compras/pt-br}.} em formato digital. A interação entre os licitantes e os agentes de contratação também se dá por via digital. Ainda é possível realizar a fase de lances e negociação de forma presencial, mas a sessão pública deve ser gravada e disponibilizada\footnote{Artigo 17.}.

Outra inovação que talvez configure o aspecto mais importante é o foco na governança. O planejamento estratégico tornou-se obrigatório por meio do Plano de Contratações Anual (PCA) e tem por objetivo alinhar as contratações às leis orçamentárias, promovendo eficiência, efetividade e eficácia \citet{TCE2022}.

O ganho de governança se dá em diversos canais:
\begin{itemize}
    \item {combate às contratações ``emergenciais'' que deveriam ter sido planejadas;}
    \item {fim da fragmentação de contratações em múltiplas contratações elegíveis para dispensa por valor;}
    \item {separação de papéis entre o agente de contratação e o adjudicador, agora sendo a autoridade superior.}
\end{itemize}

A lei divide o processo de logística pública em três partes:
\begin{enumerate}
    \item{Planejamento da Contratação;}
    \item{Seleção do Fornecedor;}
    \item{Gestão do Contrato.}
\end{enumerate}

\begin{figure}
    \centering
    \includegraphics[scale=0.26]{conteudo/imagens/fluxo.png}
    \caption{Fluxo das Contratações. Fonte: https://compras.sp.gov.br/}
    \label{fig:fluxo}
\end{figure}

\section{Planejamento da Contratação}

A fase preparatória tem início com o registro das contratações planejadas para o próximo período, consolidando-se no Plano de Contratações Anual. Além de balizar quais serão as contratações esperadas para o próximo período, subsidiando a elaboração das leis orçamentárias, o plano de contratações será divulgado e mantido à disposição do público\footnote{Artigo 12.}. Isso possibilita tanto a otimização da logística pública por meio de contratações centralizadas quanto o conhecimento pelo mercado das demandas do setor público, fomentando o diálogo entre os demandantes e os ofertantes e abrindo espaço para a inovação.

O PCA deverá levar em conta tanto as aquisições ordinárias como papel, serviço de limpeza, suprimentos de informática etc, como as aquisições esporádicas que também fizerem parte do planejamento estratégico do órgão como a atualização do parque de computadores. \citet{TCE2022}

O locus que concentra a divulgação dos planos de contratação bem como as contratações propriamente ditas é o Portal Nacional de Contratações Públicas\footnote{\url{https://www.gov.br/pncp}.}. O fato de existir um local específico simplifica o acesso à demanda do setor público para os agentes do mercado.

A demanda que constitui o plano é formada de maneira mais abstrata quando da elaboração do PCA. A etapa que efetivamente caracteriza o objeto a ser contratado é o Estudo Técnico Preliminar (ETP).

O ETP tem por objetivo \emph{evidenciar o problema a ser resolvido e sua melhor solução, de modo a permitir a avaliação da viabilidade técnica e econômica da contratação}\footnote{Artigo 18.}, trata-se de um artefato que versa principalmente sobre a necessidade da contratação.

Os elementos obrigatórios em um ETP são a descrição da necessidade da contratação, a estimativa da quantidade, a \textbf{estimativa do valor}, justificativa para o parcelamento da solução e se a contratação é viável, ou não\footnote{Pode acontecer de a contratação não ser mais interessante no momento onde ela finalmente seria executada.}. É no ETP onde há a ligação entre o PCA e o objeto a ser contratado.

Descrita a necessidade, a próxima etapa consiste em verificar quais são os riscos associados à contratação. O exemplo mais imediato é ponderar sobre a situação onde o fornecedor não entrega o material.

Completando a trinca de artefatos, há o termo de referência. É o artefato responsável pelo enlace entre o cadastro de materiais e serviços do Governo Federal e o que se deseja contratar de fato. Esse enlace é realizado na seção das condições gerais dos objetos a serem contratados.

O termo de referência é onde está descrita a especificação técnica dos itens, inclusive, com previsão de marcas e modelos que serão aceitos\footnote{Vale ressaltar que a aquisição de bens de luxo é vedada. No Decreto n{\textordmasculine} 10.818 de 2017 há a menção sobre bens de luxo serem caracterizados como bens com alta elasticidade-renda, entretanto, não se especifica o que é ``alta''.} e, interessantemente, marcas e modelos que \textbf{não serão aceitos}. Em ambos os casos deve-se justificar.

Por fim, o termo de referência também versa sobre condições garantia e aspectos de gestão contratual como habilitação dos fornecedores, execução do objeto, pagamento e adequação orçamentária.

Todos os artefatos da fase preparatória tendem a melhorar a qualidade da informação disponível para todos os envolvidos.

\begin{figure}
    \centering
    \includegraphics[scale=0.6]{conteudo/imagens/ETP.png}
    \caption{Elementos de um ETP. Os elementos obrigatórios estão nos quadrados com fundo azul. Fonte: https://www.youtube.com/watch?v=qlNwBv2MTxg.}
    \label{fig:ETP}
\end{figure}

\section{Seleção do Fornecedor}

A parte mais interessante para quem estuda mecanismos sem sombra de dúvidas é a que versa sobre a seleção de fornecedores. É nessa fase onde são empregradas as diversas modalidades bem como os instrumentos auxiliares com o intuito de realizar as contratações de bens ou serviços.

Existem dois modos de disputa: aberto e fechado. No modo aberto os licitantes apresentam suas propostas por meio de lances sucessivos. No modo fechado as propostas ficarão em sigilo até a data e hora designadas para a divulgação\footnote{Artigo 56.}.

A Lei n{\textordmasculine} 14.133 traz cinco modalidades:
\begin{enumerate}
    \item {pregão;}
    \item {concorrência;}
    \item {concurso;}
    \item {leilão;}
    \item {diálogo competitivo.}
\end{enumerate}

O pregão tende a ser a modalidade mais utilizada, pois é a modalidade obrigatória para bens e serviços comuns, inclusive de engenharia. O critério de julgamento é sempre \textbf{menor preço} ou \textbf{maior desconto}. Não é possível usar o modo fechado para o pregão, entretanto, é possível combinar ambos os modos.

A segunda modalidade mais utilizada é a concorrência \citet{TCE2023}. Trata-se da modalidade usada para obras, serviços de engenharia e serviços cuja descrição objetiva não é viável. A concorrência admite técnica e técnica e preço, além dos critérios de julgamento disponíveis para o pregão. É a única modalidade onde é possível selecionar fornecedores utilizando a técnica e preço como critério. Além disso, os critérios envolvendo técnica obrigam o uso do modo fechado. No caso de técnica e preço, a técnica pode corresponder por até 70\% da ponderação da pontuação\footnote{Artigo 36.}.

O concurso é a modalidade para a seleção de trabalho técnico, científico ou artístico cujo julgamento será necessariamente o de melhor técnica ou conteúdo artístico. O prêmio ou a remuneração do concurso são fixados já no edital de abertura do certame \citet{TCE2022}.

O leilão é a modalidade utilizada para a alienação de bens cujo critério de julgamento é o maior lance.

Por fim, o diálogo competitivo é uma inovação que prevê a divisão do certame em pelo menos duas fases. Na primeira fase o órgão público divulga a sua necessidade e os licitantes interessados submetem propostas para resolver a necessidade. Esse procedimento pode se repetir por diversas vezes. Findo o modelo, há a disputa onde só poderão participar os licitantes que fizeram parte das fases anteriores. Trata-se de uma modalidade cujo uso é restrito às contratações onde houver inovação tecnológica, impossibilidade de as especificações técnicas serem bem definidas pela Administração, impossibilidade de o órgão ter as necessidades satisfeitas com uma soluções disponíveis no mercado\footnote{Artigo 32.}.

No modo aberto é possível estabelecer via edital qual o intervalo mínimo entre os lances, embora não esteja explícito se isso será em moeda corrente ou em percentual. Ainda no modo aberto, uma vez que o vencedor tenha sido definido, é possível reiniciar a fase de lances com o objetivo de determinar a classificação de todos os licitantes.

Uma vez selecionado o vencedor, há a previsão de uma fase de negociação com o objetivo de melhorar as condições propostas. Se o preço final for superior ao preço de reserva, orçado no ETP, o licitante estará desclassificado e a negociação poderá ser realizada observando a classificação dos liciantes na etapa anterior.

Existem mais situações onde há a desclassificação de um licitante. São as seguintes:
\begin{itemize}
    \item {preço inexequível;}
    \item {inconformidade com as especificações técnicas;}
    \item {vícios \textbf{insanáveis} na proposta;}
    \item {preço final acima do preço de reserva.}
\end{itemize}

A situação de preço inexequível ocorre quando o preço final está muito abaixo do estimado. O licitante pode provar que tem condição de atender aos requisitos editalícios. Em obras de engenharia, por exemplo, é considerado inexequível um valor inferior a 75\% do orçado pela Administração\footnote{Artigo 59.}.

Há a preocupação em não desclassificar um licitante que tenha a possibilidade de sanar algum problema em sua proposta, potencialmente permitindo a participação de mais licitantes no certame. Como fato estilizado, vale dizer que existem atas processos licitatórios cuja desclassificação um licitante se deu pelo título da proposta não estar no formato solicitado no edital\footnote{Link da BEC aqui.}, diminuindo o número de concorrentes.

A Lei n{\textordmasculine} 14.133 prevê cinco procedimentos auxiliares para os processos licitatórios\footnote{Artigo 78.}. Quais sejam:
\begin{itemize}
    \item {credenciamento;}
    \item {pré-qualificação;}
    \item {procedimento de manifestação de interesse (PMI);}
    \item {registro cadastral;}
    \item {sistema de registro de preços.}
\end{itemize}

A pré-qualificação, o PMI e o registro cadastral são procedimentos que antecedem a licitação. O credenciamento e o sistema de registro de preços podem resultar na contratação de fornecedores, dispensando o procedimento licitatório \citet{TCE2022}.

O credenciamento pode ser realizado quando a contratação for paralela e não-excludente, isto é quando a contratação puder ser realizada de forma padronizada e simultânea. Quando não for possível a contratação imediata de todos os credenciados, deve ser adotado um critério de distribuição de demanda. Também é possível realizar o credenciamento quando há a situação de um mercado fluido, quando os valores variam constantemente de tal sorte que o procedimento licitatório regular se torna inviável\footnote{Artigo 79.}. Uma ocasião onde pode haver tal tipo de variação é em bens que estão sujeitos a variações cambiais \citet{TCE2022}.

A pré-qualificação é que almeja selecionar previamente bens e licitantes que atendam as exigências técnicas estabelecidas pela Administração\footnote{Artigo 80.} permitindo, posteriormente, a realização de um processo licitatório restrito aos bens e licitantes pré-qualificados. O cadastro de bens e licitantes deve ficar disponível para o público.

Tanto o credenciamento quanto a pré-qualificação são procedimentos que ficam permanentemente abertos para a inscrição de interessados.

O procedimento de manifestação de interesse é um chamamento público onde a Administração solicita à iniciativa privada a realização de estudos e propostas de soluções inovadoras para questões de relevância pública\footnote{Artigo 81.}. Ressalta-se que o PMI, diferente do diálogo competitivo, não obriga a realização de uma licitação posterior \citet{TCE2022}.

O registro cadastral é o cadastro de licitantes realizado no PNCP e atualizado anualmente. O cadastro classifica os inscritos em área de atuação, depois em categorias, segundo a qualificação técnica e econômico-financeira\footnote{Artigo 87.}.

\subsection{Sistema de Registro de Preços}
O sistema de registro de preços é definido como \emph{conjunto de procedimentos para realização, mediante contratação direta ou licitação nas modalidades pregão ou concorrência, de registro formal de preços relativos a prestação de serviços, a obras e a aquisição e locação de bens para contratações futuras}\footnote{Artigo 6.}.

O principal objetivo é evitar a realização de editais posteriores para a contratação de objetos de mesma categoria, adquirindo o que for efetivamente demandado no tempo em que houver tal demanda. O procedimento evita a aquisição desnecessária de bens bem como facilita a gestão de estoque.

Só é permitido como critério de julgamento menor preço ou maior desconto de tal sorte que o procedimento licitatório adotado geralmente será o pregão.

Podemos modelar a ata de registro de preços como um contrato onde são fornecidos o preço e a quantidade máxima de um determinado tipo de bem que \emph{pode} ser adquirido, funcionando como uma opção de compra. É possível a aquisição de \emph{zero} unidades, inclusive com a realização de outra licitação desde que devidamente justificada\footnote{Artigo 83.}.

Vale ressaltar que há algumas situações onde o preço pode ser diferente: quando o objeto for entregue em locais diferentes, em razão da forma e do local de acondicionamento, quando admitida a cotação variável em função do tamanho do lote e, finalmente, por razões justificadas no processo.

Existe a possibilidade de permitir que os licitantes registrem um quantitativo inferior ao previsto no edital bem como a possibilidade de se registrar mais de um fornecedor, desde que a cotação do objeto seja a mesma, assegurada a preferência de contratação de acordo com a ordem de classificação.

Por um lado, tal possibilidade é interessante, pois o perfil de custo marginal dos licitantes tipicamente não é constante, ao contrário do que acontece num contrato de registro de preços. Isso significa que se for vantajoso para um fornecedor entregar um quantitativo inferior ao solicitado, ele poderá participar do certame, aumentando potencialmente a competição. Por outro, há a possibilidade de divisão de lote, elevando o preço final\footnote{Basta que o primeiro colocado não oferte todo o quantitativo. Em vez de dar um lance para ganhar do segundo colocado, ele poderia dar um lance para ganhar apenas do terceiro colocado, majorando o lucro pela prática de um preço superior.}.

Uma inovação do sistema de registro de preços na nova lei versa sobre a possibilidade do registro de preço ser por grupo de itens desde que haja vantagem técnica ou econômica. Tal inovação ajuda a mitigar situações de exposição, onde há complementaridade entre bens, pois o caráter linear do registro de preços permite, em tese, quebrar essa sinergia\footnote{Um exemplo seria a aquisição de sapato por unidades de pé esquerdo e de pé direito. Um exemplo melhor seria a aquisição de peças de computador de uma mesma arquitetura.}.

Outra inovação versa sobre a participação de entidades não originalmente participantes, o que se convencionou a chamar de \emph{carona}. Quando da fase preparatória, há a obrigação de se realizar um procedimento público de intenção por, no mínimo, 8 dias para possibilitar a participação de outros órgãos na ata. Somente depois da adesão é que se calcula, de fato, o quantitativo.

Na nova lei há a possibilidade de adesão de órgãos que não participaram da fase preparatória e nem da fase de adesão desde que haja justificativa sobre a vantagem da adesão, a demonstração de que os valores registrados estão compatíveis com o mercado e a anuência do órgão gerenciador e do fornecedor. Contudo a adesão só pode realizada na direção do órgão mais restrito para o mais geral, isto é, órgãos federais não podem aderir a registros de preço estaduais e órgãos estaduais não podem aderir a registros de preço de órgãos municipais\footnote{Artigo 86.}.

Há duas limitações de quantitativo para a adesão de caronas nas atas de registro de preço: no máximo 50\% do quantitativo por carona e um limite global de duas vezes o quantitativo original para todos os caronas. A única exceção é caso se tratar de aquisição emergencial em ata gerenciada pelo Ministério da Saúde.

\begin{figure}
    \centering
    \includegraphics[scale=0.35]{conteudo/imagens/srp-adesao.png}
    \caption{A adesão não pode ser realizada de órgão mais geral para órgão mais restrito. Fonte: \citet{TCE2023}}
    \label{fig:srp-adesao}
\end{figure}

Na nova lei é possível haver registro de preço para serviços comuns de engenharia desde que exista projeto padronizado, sem complexidade técnica e operacional e que os serviços contratados sejam necessidade constante da Administração\footnote{Artigo 85.}. Um exemplo de serviço dessa natureza é a instalação de pontos de rede estruturada de dados. É um serviço padronizado, sem complexidade e demandado regularmente.
%!TeX root=../projeto.tex
%("dica" para o editor de texto: este arquivo é parte de um documento maior)
% para saber mais: https://tex.stackexchange.com/q/78101

%% ------------------------------------------------------------------------- %%

% "\chapter" cria um capítulo com número e o coloca no sumário; "\chapter*"
% cria um capítulo sem número e não o coloca no sumário. A introdução não
% deve ser numerada, mas deve aparecer no sumário. Por conta disso, este
% modelo define o comando "\unnumberedchapter".
\chapter{Problema}
\label{cap:problema}

Pretende-se modelar algumas situações que envolvem pregões num espírito parecido com \citet{Bugarin2022}. A primeira diferença é que um pregão regular os bens são adquiridos como parte do processo enquanto que num pregão por registro de preços o que é ``adquirido'' é um contrato que prevê o direito de o órgão público adquirir os bens que estão no contrato pelo preço definido no certame, isto é, há a possibilidade de se adquirir \emph{zero} unidades.

Tal diferença induz o modelo a adicionar uma ação para o pregoeiro: a possibilidade de adquirir um quantitativo menor que o quantitativo previsto no edital. Isso sugere a possibilidade de ambos os processos apresentarem equilíbrios diferentes.

\section{Pregão por registro de preço}

A hipótese simplificadora será a de que o órgão público demandará apenas um único bem, assim como no modelo apresentado por \citet{Bugarin2022}. Contudo, como na Lei n{\textordmasculine} 14.133 só admite menor preço ou maior desconto, o modelo de pregão é o de lances sucessivos e, portanto, melhor modelado por um pregão de segundo menor preço.

O modelo de um pregão assume $N$ licitantes simétricos cuja ação prevista é a de dar um lance com o objetivo de maximizar seus \emph{payoffs}. Em caso de vitória, o \emph{payoff} é dado pela diferença entre o lance e o custo, que é o valor privado independente. Caso contrário, o \emph{payoff} será zero. Além disso, o pregoeiro pode demandar o bem, ou não, seguindo uma distribuição de probabilidade conhecida por todos os jogadores.

Aqui existem duas possibilidades de modelagem sobre a aquisição do bem. Ela pode ser:
\begin{enumerate}
	\item independente do preço;
	\item dependente do preço.
\end{enumerate}

Uma justificativa para o primeiro caso é o órgão público possuir renda o suficiente para adquirir o bem e a decisão de aquisição ser pautada por alguma razão interna ao órgão. Uma possibilidade seria a aquisição de computadores condicional à vinda de funcionários. Esse evento não possui relação com o lance dos jogadores e determina integralmente a aquisição.

No segundo caso pode haver um certo raciocínio de custo-benefício a partir do órgão público. Nesse sentido, quanto menor o preço, maior a probabilidade de o bem ser adquirido.

Contudo, por se tratar de um pregão de segundo preço, o segundo preço permanece sendo a valoração do segundo colocado, pois $\beta(x) = x$ continua sendo o menor lance possível para ele.

Por outro lado, se fosse num pregão de primeiro preço

\section{Pregão por registro de preço com qualidade}

%%%%%%%%%%%%%%%%%%%%%%%%%%%% APÊNDICES E ANEXOS %%%%%%%%%%%%%%%%%%%%%%%%%%%%%%%%

%%%% Apêndices %%%%

%\makeatletter
\if@openright\cleardoublepage\else\clearpage\fi
\makeatother

\pagestyle{appendix}

\appendix

% \addappheadtotoc acrescenta a palavra "Apêndice" ao sumário; se
% só há apêndices, sem anexos, provavelmente não é necessário.
%\addappheadtotoc

\chapter{Demonstrações}
\label{cap:apendice}

\section{Estratégia dominante no leilão de segundo preço}
\begin{proof}[Prova da Proposição \ref{prop:estrategia-dominante}]
	Vamos olhar apenas para as situações onde o $i$ pode ganhar o leilão.
	
	Se $i$ desviar e jogar $y_i > x_i$, um jogador $j$ pode jogar $y_j : y_i > y_j > x_i$. Nesse caso, $i$ ganha o leilão, mas pagará $y_j$, ficando com $x_i - y_j < 0$. Se tivesse jogado $x_i$, perderia 0.
	
	Se $i$ desviar e jogar $z_i < x_i$, um jogador $j$ pode jogar $z_j : x_i > z_j > z_i$. Nesse caso, $i$ perde o leilão, mas poderia ter ganho $x_i - z_j > 0$.
	
	Portanto $\beta(x_i) = x_i$.
\end{proof}

\section{Lance no leilão de primeiro preço}
\begin{proof}[Prova da Proposição \ref{prop:nash-primeiro-preco}]
	Como todos os jogadores implementam a melhor resposta em um equilíbrio de Nash, vamos maximizar para $b$. Por simplicidade, vamos remover o $i$ subscrito.
	\begin{align*}
		\frac{d G(\beta^{-1}(b)) (x - b)}{db} = 0 \Rightarrow
	\end{align*}
	\begin{align*}
		\frac{ g( \beta^{-1}(b) ) }{ \beta'( \beta^{-1}(b) ) }(x - b) - G( \beta^{-1}(b) ) = 0 \Rightarrow
	\end{align*}
	\begin{align*}
		G( \beta^{-1}(b) )\beta'( {\beta^{-1}(b)} ) = g( \beta^{-1}(b) )(x - b) \Rightarrow
	\end{align*}
	\begin{align*}
		g(\beta^{-1}(b))b + G(\beta^{-1}(b))\beta'({\beta^{-1}(b)}) = g(\beta^{-1}(b))x
	\end{align*}

	Aplicando a hipótese de simetria, $b = \beta(x)$, segue que:
	\begin{align*}
		g(x)\beta(x) + G(x)\beta'(x) & = xg(x) \Rightarrow
	\end{align*}
	\begin{align*}
		\frac{d}{dx} \beta(x)G(x) & = xg(x) \Rightarrow
	\end{align*}
	\begin{align*}
		\left[ \beta(x)G(x) \right]_{0}^{x} & = \int_{0}^{x} yg(y)dy \Rightarrow
	\end{align*}
	\begin{align*}
		\beta(x)G(x) - \cancelto{0}{\beta(0)G(0)} & = \int_{0}^{x} yg(y)dy \Rightarrow
	\end{align*}
	\begin{align*}
		\beta(x) = \frac{1}{G(x)} \int_{0}^{x} yg(y)dy
	\end{align*}
\end{proof}

\section{Lance no leilão de primeiro preço com opção de aquisição}
\begin{proof}[Prova da Proposição \ref{prop:nash-primeiro-preco-opcao}]
	Como todos os jogadores implementam a melhor resposta em um equilíbrio de Nash, vamos maximizar para $b$.
	\begin{align*}
		& \frac{d H(\beta^{-1}(b)) G(\beta^{-1}(b)) (x - b)}{db} = 0 \Rightarrow
	\end{align*}
	\begin{align*}
		\frac { h(\bbeta}{\dinvbeta} \left[ G(\bbeta)(x - b) \right] & + H(\bbeta) \left[ \frac{g(\bbeta)}{\dinvbeta}(x - b) - G(\bbeta) \right] = 0 \Rightarrow
	\end{align*}
	\begin{align*}
		h(\bbeta) \left[ G(\bbeta)(x - b) \right] & + H(\bbeta) \left[ g(\bbeta)(x - b) \right] = H(\bbeta)G(\bbeta)\dinvbeta
	\end{align*}
	Aplicando a hipótese de simetria, $b = \beta(x)$, segue que:
	\begin{align*}
		h(x)G(x)(x-\beta(x)) + H(x)g(x)(x-\beta(x)) = H(x)G(x)\beta'(x) \Rightarrow
	\end{align*}
	\begin{align*}
		\left[ h(x)G(x) + H(x)g(x) \right]x = \left[ h(x)G(x) + H(x)g(x) \right]\beta(x) + H(x)G(x)\beta'(x) \Rightarrow
	\end{align*}
	\begin{align*}
		\frac {d H(x)G(x) }{dx} x & = \frac{d H(x)G(x)\beta(x)}{dx} \Rightarrow
	\end{align*}
	\begin{align*}
		\int_0^x \frac {d H(y)G(y) }{dy} ydy & = \int_0^x \frac{d H(y)G(y)\beta(y)}{dy}dy \Rightarrow
	\end{align*}
	\begin{align*}
    	\int_0^x (H(y)G(y))'ydy & = H(x)G(x)\beta(x) - \cancelto{0}{H(0)G(0)\beta(0)} \Rightarrow
    \end{align*}
	\begin{align*}
    	\left[ H(y)G(y)y \right]_0^x - \int_0^x H(y)G(y)dy & = H(x)G(x)\beta(x) \Rightarrow
    \end{align*}
	\begin{align*}
    	H(x)G(x)x - \int_0^x H(y)G(y)dy & = H(x)G(x)\beta(x) \Rightarrow
	\end{align*}
	\begin{align*}
    	\beta(x) = x - \int_0^x \frac{H(y)G(y)}{H(x)G(x)}dy
	\end{align*}
\end{proof}

\section{Lance no leilão de segundo preço com opção de aquisição}

\subsection{Primeira parte: encontrando a equação integral}
\begin{lema}
	A estratégia de equilíbrio simétrico num leilão de segundo preço como a opção de não se adquirir o bem é dada por:
	\begin{equation}
		\beta(x) = x + h(x)\int_0^x \frac{\beta'(y)G(y)dy}{G(x)h(x) + H(x)g(x)}
	\end{equation}
\end{lema}
\begin{proof}
	A diferença entre os leilões de primeiro e de segundo preço é de natureza de \emph{payoff}, além disso, valem as condições para o Teorema da Equivalência de Receita.
	Seja $m(x)$ o pagamento no leilão de primeiro preço:
	\begin{align*}
		m(x) = \int_0^x y(G(y)h(y)+H(y)g(y)) dy
	\end{align*}
	Por outro lado, considerando $Y$ a estatística de maior ordem dos outros $n-1$ participantes, o pagamento no leilão de segundo preço será:
	\begin{align*}
		m(x) = H(x)G(x)E[\beta(Y) \mid Y<x] = H(x) \int_0^x \beta(y)g(y)dy
	\end{align*}
	Igualando os pagamentos e derivando em relação a $x$:
	\begin{align*}
		\frac{d}{dx} \left[ H(x) \int_0^x \beta(y)g(y)dy \right] = \frac{d}{dx} \int_0^x y(G(y)h(y)+H(y)g(y)) dy & \Rightarrow
	\end{align*}
	\begin{align*}
		h(x) \int_0^x \beta(y)g(y)dy + H(x)\beta(x)g(x) = x\left[ G(x)h(x) + H(x)g(x) \right] & \Rightarrow
	\end{align*}
	\begin{align*}
		h(x) \left[ \left[ \beta(x)G(x) \right]_0^x - \int_0^x \beta'(y)G(y)dy \right] + H(x)\beta(x)g(x) & = x\left[ G(x)h(x) + H(x)g(x) \right] & \Rightarrow
	\end{align*}
	\begin{align*}
		\left[ G(x)h(x) + H(x)g(x) \right]\beta(x) - h(x) \int_0^x \beta'(y)G(y)dy & = x\left[ G(x)h(x) + H(x)g(x) \right] & \Rightarrow
	\end{align*}
	\begin{align*}
		\beta(x) = x + h(x) \int_0^x \frac{\beta'(y)G(y)}{G(x)h(x) + H(x)g(x)}dy
	\end{align*}
\end{proof}

\subsection{Segunda parte: resolvendo a equação integral}
\begin{proof}[Prova da Proposição \ref{prop:nash-segundo-preco-opcao}]
	\begin{align*}
		\beta(x) = x + \frac{h(x)}{G(x)h(x) + H(x)g(x)} \int_0^x \beta'(y)G(y)dy \Rightarrow
	\end{align*}
	\begin{align*}
		\left[ \beta(x) - x \right] \frac{G(x)h(x) + H(x)g(x)}{h(x)} = \int_0^x \beta'(y)G(y)dy \Rightarrow
	\end{align*}
	\begin{align*}
		\left[ \beta(x) - x \right] \frac{G(x)h(x) + H(x)g(x)}{h(x)} = \left[ \beta(y)G(y) \right]_0^x - \int_0^x \beta(y)g(y)dy \Rightarrow
	\end{align*}
	\begin{align*}
		-x \frac{G(x)h(x) + H(x)g(x)}{h(x)} = \beta(x) \left[ G(x) - \frac{G(x)h(x) + H(x)g(x)}{h(x)} \right] - \int_0^x \beta(y)g(y)dy \Rightarrow
	\end{align*}
	\begin{align*}
		-x \frac{G(x)h(x) + H(x)g(x)}{h(x)} = \beta(x) \left[ \frac{\cancel{G(x)h(x)} - \cancel{G(x)h(x)} - H(x)g(x)}{h(x)} \right] - \int_0^x \beta(y)g(y)dy \Rightarrow
	\end{align*}
	\begin{align*}
		x \frac{G(x)h(x) + H(x)g(x)}{h(x)} = \beta(x) \left[ \frac{H(x)g(x)}{h(x)} \right] + \int_0^x \beta(y)g(y)dy \Rightarrow
	\end{align*}
	\begin{align*}
		x \frac{G(x)h(x) + H(x)g(x)}{\cancel{h(x)}}\frac{\cancel{h(x)}}{H(x)g(x)} = \beta(x) + \frac{h(x)}{H(x)g(x)}\int_0^x \beta(y)g(y)dy \Rightarrow
	\end{align*}
	\begin{align*}
		x + x \frac{G(x)h(x)}{H(x)g(x)} = \beta(x) + \frac{h(x)}{H(x)g(x)}\int_0^x \beta(y)g(y)dy
	\end{align*}
	A equação integral está no mesmo formato da equação 2.9.2 em \citet{polyanin1998handbook}:
	\begin{equation*}
		f(x) = y(x) - \int_a^x \widehat{g}(x)\widehat{h}(t)y(t)dt
	\end{equation*}
	Onde:
	\begin{align*}
		y(x) & = \beta(x) \\
		f(x) & = x + x \frac{G(x)h(x)}{H(x)g(x)} \\
		\widehat{g}(x) & = -\frac{h(x)}{H(x)g(x)} \\
		\widehat{h}(t) & = g(y) \\
		dt &= dy \\
		a &= 0
	\end{align*}
	A solução será dada por:
	\begin{equation*}
		y(x) = f(x) + \int_a^x R(x,t)f(t)dt \text{, onde } R(x,t) = \widehat{g}(x)\widehat{h}(t)exp\left[ \int_t^x g(s)h(s)ds \right]
	\end{equation*}
	Trocaremos o $y$ por $t$, para combinar com a notação do livro. Vamos calcular $R(x,t)$:
	\begin{align*}
		R(x,t) & = -\frac{h(x)}{H(x)g(x)} g(t) exp\left[ \int_t^x -\frac{h(s)}{H(s)g(s)}g(s)ds \right] \Rightarrow \\
		& = -\frac{h(x)}{H(x)g(x)} g(t) exp\left[ \int_t^x -\frac{h(s)}{H(s)}ds \right] \Rightarrow \\
		& = -\frac{h(x)}{H(x)g(x)} g(t) exp\left[ ln \left( \frac{H(t)}{H(x)} \right) \right] \Rightarrow \\
		& = -\frac{h(x)}{H(x)g(x)} g(t) \frac{H(t)}{H(x)} \Rightarrow \\
		& = -\frac{h(x)g(t)H(t)}{H(x)^2g(x)}
	\end{align*}
	Dado $R(x,t)$, procederemos com o cálculo da integral:
	\begin{align*}
		\int_0^x R(x,t)f(t)dt & = \int_0^x -\frac{h(x)g(t)H(t)}{H(x)^2g(x)} \frac{[G(t)h(t)+H(t)g(t)]t}{H(t)g(t)} dt \Rightarrow \\
		& = \int_0^x -\frac{h(x)\cancel{g(t)}\cancel{H(t)}}{H(x)^2g(x)} \frac{[G(t)h(t)+H(t)g(t)]t}{\cancel{H(t)}\cancel{g(t)}} dt \Rightarrow \\
		& = -\frac{h(x)}{H(x)^2g(x)} \int_0^x  t \left[ G(t)h(t)+H(t)g(t) \right] dt \Rightarrow \\
		& = -\frac{h(x)}{H(x)^2g(x)} \left[ tG(t)H(t) \right]_0^x -\int_0^x G(t)H(t)dt \Rightarrow \\
		& = -x \frac{h(x)G(x)\cancel{H(x)}}{H(x)^{\cancel{2}}g(x)} + \frac{h(x)}{H(x)^2g(x)}\int_0^x G(t)H(t)dt
	\end{align*}
	A solução final é dada por:
	\begin{align*}
		\beta(x) & = x + \cancel{x \frac{G(x)h(x)}{H(x)g(x)}} - \cancel{x \frac{h(x)G(x)}{H(x)g(x)}} + \frac{h(x)}{H(x)^2g(x)}\int_0^x G(t)H(t)dt \Rightarrow \\
		& = x + \frac{h(x)}{H(x)^2g(x)}\int_0^x G(t)H(t)dt
	\end{align*}
\end{proof}

\section{Lance no pregão de segundo preço por registro de preços com qualidade}

\subsection{Primeira parte: encontrando a equação integral}
\begin{lema}
	\label{lema:nash-pregao-segundo-preco-qualidade}
	A estratégia de equilíbrio simétrico num pregão de segundo preço por registro de preços com recusa por qualidade é dada por:
	\begin{equation}
		\beta(\theta) = c(q, \theta) + \frac{h(q)q'(\theta)}{H(q)g(\theta)-h(q)q'(\theta)[1-G(\theta)]} \int_{\theta}^1 \beta'(y)[1-G(y)]dy
	\end{equation}
\end{lema}
\begin{proof}
	O objetivo, como usual, é maximizar o \emph{payoff}, que é dado por:
	\begin{align*}
		P[aceitar]P[vencer](E[\beta(Y) \mid Y > b]-c(q,\theta)) = 
	\end{align*}
	\begin{align*}
		H(q)[1-G(\bbeta)] \left[ \int_{\bbeta}^1 \frac{\beta(y)g(y)}{1-G(\bbeta)}dy - c(\theta) \right] \Rightarrow
	\end{align*}
	\begin{align*}
		H(q) \left[ \int_{\bbeta}^1 \beta(y)g(y)dy - [1-G(\bbeta)]c(\theta) \right]
	\end{align*}
	Maximizando para $b$:
	\begin{align*}
		H(q) \left[ -\frac{\beta(\bbeta)g(\bbeta)}{\cancel{\dinvbeta}} + \frac{g(\bbeta)}{\cancel{\dinvbeta}}c(\theta) \right] +
	\end{align*}
	\begin{align*}
		\frac{h(q)q'(\bbeta)}{\cancel{\dinvbeta}} \left[ \int_{\bbeta}^1 \beta(y)g(y)dy - [1-G(\bbeta)]c(\theta) \right] = 0
	\end{align*}
	Aplicando a hipótese de equilíbrio $\beta(\theta) = b$:
	\begin{align*}
		H(q) \left[ -\beta(\theta)g(\theta) + g(\theta)c(\theta) \right] + h(q)q'(\theta) \left[ \int_{\theta}^1 \beta(y)g(y)dy - [1-G(\theta)]c(\theta) \right] = 0 \Rightarrow
	\end{align*}
	\begin{align*}
		H(q)g(\theta)c(\theta)-h(q)q'(\theta)[1-G(\theta)]c(\theta) + h(q)q'(\theta) \int_{\theta}^1 \beta(y)g(y)dy = H(q)g(\theta)\beta(\theta) \Rightarrow
	\end{align*}
	\begin{align*}
		c(\theta)[H(q)g(\theta)-h(q)q'(\theta)[1-G(\theta)]] +
	\end{align*}
	\begin{align*} h(q)q'(\theta) \left[ [-\beta(y)[1-G(y)]]_{\theta}^1 + \int_{\theta}^1 \beta'(y)[1-G(y)]dy \right] = H(q)g(\theta)\beta(\theta) \Rightarrow
	\end{align*}
	\begin{align*}
		c(\theta)[H(q)g(\theta)-h(q)q'(\theta)[1-G(\theta)]] + h(q)q'(\theta) \int_{\theta}^1 \beta'(y)[1-G(y)]dy =
	\end{align*}
 	\begin{align*}
		\beta(\theta)[H(q)g(\theta) - h(q)q'(\theta)[1-G(\theta)]] \Rightarrow
	\end{align*}
	\begin{align*}
		\beta(\theta) = c(q,\theta) + \frac{h(q)q'(\theta)}{H(q)g(\theta) - h(q)q'(\theta)[1-G(\theta)]} \int_{\theta}^1 \beta'(y)[1-G(y)]dy
	\end{align*}
\end{proof}
\par

%%%%%%%%%%%%%%% SEÇÕES FINAIS (BIBLIOGRAFIA E ÍNDICE REMISSIVO) %%%%%%%%%%%%%%%%

% O comando backmatter desabilita a numeração de capítulos.
\backmatter

\pagestyle{backmatter}

% Espaço adicional no sumário antes das referências / índice remissivo
\addtocontents{toc}{\vspace{2\baselineskip plus .5\baselineskip minus .5\baselineskip}}

% A bibliografia é obrigatória

%\nocite{*}
\printbibliography[
  title=\refname\label{bibliografia}, % "Referências", recomendado pela ABNT
  %title=\bibname\label{bibliografia}, % "Bibliografia"
  heading=bibintoc, % Inclui a bibliografia no sumário
]

\end{document}
