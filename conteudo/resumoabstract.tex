% As palavras-chave são obrigatórias, em português e em inglês, e devem ser
% definidas antes do resumo/abstract. Acrescente quantas forem necessárias.
\palavrachave{leilão}
\palavrachave{licitação}
\palavrachave{Sistema de Registro de Preços}

\keyword{auction}
\keyword{procurement}
\keyword{Sistema de Registro de Preços}

% O resumo é obrigatório, em português e inglês. Estes comandos também
% geram automaticamente a referência para o próprio documento, conforme
% as normas sugeridas da USP.
\resumo{
Este trabalho estuda a Lei de Licitações e especialmente o Sistema de Registro de Preços sob a ótica da Teoria dos Leilões. Desenvolve-se duas extensões para os pregões: uma onde o órgão público pode não comprar o bem contratado considerando o seu preço depois do certame e outra que leva em consideração a qualidade nesse mesmo arcabouço de opção de compra. Na primeira situação, os lances são mais agressivos que se comparados à situação regular. Na segunda, a qualidade ofertada tende a ser maior que a qualidade mínima.
}

\abstract{
This work studies the brazilian's procurement law and the Sistema de Registro de Preços, a contract similar to a purchase option, using Auction Theory framework. Two extensions to the procurement process are developed: one where the public body has the option to not purchase the good considering its price after the procurement and other where the quality problem is addressed using the same framework. In the first situation, the bids are more aggressive than in the regular situation. In the second, the quality offered tends to be higher than the minimum allowed quality.
}