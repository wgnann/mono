\chapter{Observações e Conclusões}
\label{cap:conclusoes}

\section{Reflexões sobre a Lei n\textordmasculine 14.133}
Talvez o melhor jeito de um leiloeiro se proteger durante um certame é criar condições para a participação da maior quantidade de jogadores. Isso tende a diminuir os preços por elevar o nível de competição.

A nova lei é potencialmente exitosa nessa direção por minorar os entraves de natureza burocrática, permitindo a possibilidade de se sanar vícios de edital bem como aplicando, por padrão, o procedimento de inversão de fases, onde a habilitação ocorre posteriormente à definição do preço.

Em termos de conluio é evidente que um número reduzido de participantes diminui o custo de comunicação para o êxito bem como diminui o montante a ser repartido pelo conluio \citet{pellegrini2018:MSc}.

\section{Sobre este trabalho}

Talvez a primeira observação que se deva fazer no presente trabalho é que a revisão bibliográfica, se abordou, meramente tangenciou os casos onde as propriedades de um leilão \emph{padrão} são violadas. A bem da verdade, como os modelos para registro de preços são iniciais, sua construção se deu no mundo onde \emph{tudo dá certo}, o mundo do \emph{abridor de lata}\footnote{O abridor de lata é o mesmo abridor de lata da anedota do físico, do químico e do economista, quando deparados com um carregamento de comida enlatada numa ilha deserta\ldots \emph{Considere um abridor de lata}, logo abrir a lata é trivial.}.

Nesse sentido um trabalho futuro evidente é estender os modelos para situações onde tais propriedades não funcionem mais como, por exemplo, quando os agentes não são mais simétricos ou quando a aversão ao risco existe. Outro trabalho futuro igualmente evidente é contrastar os modelos com os dados disponíveis sobre registro de preços tanto na Bolsa Eletrônica de Compras do Estado de São Paulo quanto do Compras.gov.br.

Outra possibilidade futura é tratar o caso multi-objeto. É bastante comum que os registros de preço se apresentem tanto na forma de um único lote quanto em lotes separados. A lei de licitações sugere que se analise durante a fase de planejamento se é viável a divisão do objeto em múltiplos lotes. Como o registro de preço se comporta de maneira linear, isto é, com custo marginal constante e com a possibilidade de se adquirir quaisquer quantitativos abaixo do devidamente contratado, há uma nova fonte de risco para os licitantes. Será que os licitantes se protegem quando há sinergia/anti-sinergia entre os itens que estão na ata de registro de preços?