\chapter{Problema}
\label{cap:problema}

Pretende-se modelar algumas situações que envolvem pregões num espírito parecido com \citet{Bugarin2022}. A primeira diferença é que um pregão regular os bens são adquiridos como parte do processo enquanto que num pregão por registro de preços o que é ``adquirido'' é um contrato que prevê o direito de o órgão público adquirir os bens que estão no contrato pelo preço definido no certame, isto é, há a possibilidade de se adquirir \emph{zero} unidades.

Tal diferença induz o modelo a adicionar uma ação para o pregoeiro: a possibilidade de adquirir um quantitativo menor que o quantitativo previsto no edital. Isso sugere a possibilidade de ambos os processos apresentarem equilíbrios diferentes.

\section{Pregão por registro de preço}

A hipótese simplificadora será a de que o órgão público demandará apenas um único bem, assim como no modelo apresentado por \citet{Bugarin2022}. Contudo, como na Lei n{\textordmasculine} 14.133 só admite menor preço ou maior desconto, o modelo de pregão é o de lances sucessivos e, portanto, melhor modelado por um pregão de segundo menor preço.

O modelo de um pregão assume $N$ licitantes simétricos cuja ação prevista é a de dar um lance com o objetivo de maximizar seus \emph{payoffs}. Em caso de vitória, o \emph{payoff} é dado pela diferença entre o lance e o custo, que é o valor privado independente. Caso contrário, o \emph{payoff} será zero. Além disso, o pregoeiro pode demandar o bem, ou não, seguindo uma distribuição de probabilidade conhecida por todos os jogadores.

Aqui existem duas possibilidades de modelagem sobre a aquisição do bem. Ela pode ser:
\begin{enumerate}
	\item independente do preço;
	\item dependente do preço.
\end{enumerate}

Uma justificativa para o primeiro caso é o órgão público possuir renda o suficiente para adquirir o bem e a decisão de aquisição ser pautada por alguma razão interna ao órgão. Uma possibilidade seria a aquisição de computadores condicional à vinda de funcionários. Esse evento não possui relação com o lance dos jogadores e determina integralmente a aquisição.

No segundo caso pode haver um certo raciocínio de custo-benefício a partir do órgão público. Nesse sentido, quanto menor o preço, maior a probabilidade de o bem ser adquirido.

Contudo, por se tratar de um pregão de segundo preço, o segundo preço permanece sendo a valoração do segundo colocado, pois $\beta(x) = x$ continua sendo o menor lance possível para ele.

Por outro lado, se fosse num pregão de primeiro preço

\section{Pregão por registro de preço com qualidade}