\chapter{Modelo}
\label{cap:modelo}

Pretende-se modelar algumas situações que envolvem pregões num espírito parecido com \citet{Bugarin2022}. A primeira diferença é que um pregão regular os bens são adquiridos como parte do processo enquanto que num pregão por registro de preços o que é ``adquirido'' é um contrato que prevê o direito de o órgão público adquirir os bens que estão no contrato pelo preço definido no certame, isto é, há a possibilidade de se adquirir \emph{zero} unidades.

Tal diferença induz o modelo a adicionar uma ação para o pregoeiro: a possibilidade de adquirir um quantitativo menor que o quantitativo previsto no edital. Isso sugere a possibilidade de ambos os processos apresentarem equilíbrios diferentes.

\section{Pregão por registro de preço}

A hipótese simplificadora será a de que o órgão público demandará apenas um único bem, assim como no modelo apresentado por \citet{Bugarin2022}. Contudo, como na Lei n{\textordmasculine} 14.133 só admite menor preço ou maior desconto, o modelo de pregão é o de lances sucessivos e, portanto, melhor modelado por um pregão de segundo menor preço.

O modelo de um pregão assume $N$ licitantes simétricos cuja ação prevista é a de dar um lance com o objetivo de maximizar o \emph{payoff}. Em caso de vitória, ganha-se diferença entre o lance e o custo, valor privado independente. Caso contrário, o \emph{payoff} será zero. Além disso, o pregoeiro pode demandar o bem, ou não, seguindo uma distribuição de probabilidade conhecida por todos os jogadores.

Aqui existem duas possibilidades de modelagem sobre a aquisição do bem. Ela pode ser:
\begin{enumerate}
	\item independente do preço;
	\item dependente do preço.
\end{enumerate}

Uma justificativa para o primeiro caso é o órgão público possuir renda o suficiente para adquirir o bem e a decisão de aquisição ser pautada por alguma razão interna ao órgão. Uma possibilidade seria a aquisição de computadores condicional à vinda de funcionários. Esse evento não possui relação com o lance dos jogadores e determina integralmente a aquisição.

No segundo caso pode haver um certo raciocínio de custo-benefício a partir do órgão público. Nesse sentido, quanto menor o preço, maior a probabilidade de o bem ser adquirido.

Analisaremos o caso dual com o objetivo de compreender o comportamento de equilíbrio.

\section{Leilão de primeiro preço com opção de aquisição}

Um leilão de primeiro preço com opção de aquisição é um leilão como visto no capítulo \ref{cap:revisao} com o adicional de que o vendedor pode decidir \textbf{não} vender o bem considerando o lance ganhador. Uma forma de interpretar é imaginar que há um preço de reserva oculto \citet{Bugarin2022}. Outra interpretação é que o único jeito, no mundo restrito do modelo, de o participante sinalizar para o leiloeiro seu interesse no bem é pelo lance, embora esse lance possa conter informação de outras naturezas como o \emph{score} de crédito do participante ou alguma medida de credibilidade.

Seja $H()$ a função de probabilidade acumulada do leiloeiro, o \emph{payoff} esperado será determinado por:
\begin{align*}
	P[vender]P[vencer](x_i - b) \Rightarrow H(\beta^{-1}(b))G(\beta^{-1}(b))(x_i - b)
\end{align*}

\begin{proposicao}
	\label{prop:nash-primeiro-preco-opcao}
	A estratégia de equilíbrio simétrico num leilão de primeiro preço com a opção de não se adquirir o bem é dada por:
	\begin{equation}
		\beta(x) = x - \int_0^x \frac{G(y)H(y)}{G(x)H(x)}dy
	\end{equation}
\end{proposicao}

O pagamento esperado realizado pelo vencedor será:
\begin{equation}
	H(x)G(x) \left[ x - \int_0^x \frac{G(y)H(y)}{G(x)H(x)}dx \right] = \int_0^x y(G(y)h(y)+H(y)g(y)) dy
\end{equation}

\section{Leilão de segundo preço com opção de aquisição}

\begin{proposicao}
	\label{prop:nash-segundo-preco-opcao}
	A estratégia de equilíbrio simétrico num leilão de segundo preço como a opção de não se adquirir o bem é dada por:
	\begin{equation}
		\beta(x) = x + \frac{h(x)}{H(x)^2 g(x)} \int_0^x G(y)H(y)dy
	\end{equation}
\end{proposicao}

\section{Pregão de segundo preço por registro de preços}

\begin{proposicao}
	\label{prop:nash-pregao-segundo-preco-rp}
	A estratégia de equilíbrio simétrico num pregão de segundo preço por registro de preços é dada por:
	\begin{equation}
		\beta(x) = x - \frac{h(x)}{[1-H(x)]^2 g(x)} \int_x^1 [1-G(y)][1-H(y)]dy
	\end{equation}
\end{proposicao}
\begin{proof}
	Trata-se do caso dual do resultado da proposição \ref{prop:nash-segundo-preco-opcao}.
\end{proof}

\section{Pregão de segundo preço por registro de preços com qualidade}

\begin{lema}
	Com leilões (reversos) de primeiro e de segundo preço, a qualidade é escolhida em $q_s(\theta)$ para $\theta \in [0, 1]$ onde:
	\begin{equation}
		q_s(\theta) = \text{argmax } s(q) - c(q, \theta)
	\end{equation}
	\begin{proof}
		Em \citet{Che1993}.
	\end{proof}
\end{lema}

\begin{proposicao}
	\label{prop:nash-pregao-segundo-preco-qualidade}
	A estratégia de equilíbrio simétrico num pregão de segundo preço por registro de preços com qualidade é dada por:
	\begin{equation}
		\beta(q, \theta) = c(q, \theta) + \frac{h(q)q'(\theta)}{H(q)^2g(\theta)}\int_{\theta}^1 c_{y}(q, y)[1-G(y)]H(q(y))dy
	\end{equation}
\end{proposicao}