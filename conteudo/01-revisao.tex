\chapter{Revisão bibliográfica}
\label{cap:revisao}

\section{Leilões}

\emph{Dou-lhe uma! Dou-lhe duas! Dou-lhe três!} O leilão certamente figura entre as instituições mais antigas e mais conhecidas da história da humanidade. Situa-se, na Idade Antiga, desde 500a.C., na Babilônia, onde as famílias leiloavam as filhas para casamento, passando pelo leilão do Império Romano em 193 a.C., vencido por Didius Julianus\footnote{Talvez não tenha sido um bom investimento, pois Didius Julianus morreu dois meses depois, decapitado pelas legiões lideradas por Septimius Severus. Segundo \citet{krishna}, trata-se de um caso extremo da maldição do vencedor.} ao dar a maior doação \citet{Cassady2021-ac}.

Também é bastante presente na cultura popular\footnote{\url{https://tvtropes.org/pmwiki/pmwiki.php/Main/Auction}.} fazendo parte de séries de TV como \emph{Star Trek}, jogos de videogame como \emph{Diablo} e \emph{Final Fantasy}, animações como \emph{WiFi Ralph} e \emph{Futurama} e músicas como \emph{Leilão}\footnote{Glória Groove. Clipe em: \url{https://www.youtube.co/watch?v=X7cz9v6mHzg}.}. Não é um exagero dizer que o senso comum entende o conceito de leilão e que esse conceito tem mais de 2000 anos.

Hoje em dia os leilões são usados para diversas finalidades: venda de lotes de itens da Receita Federal, venda de ativos financeiros, venda de propaganda em sites da Internet\footnote{A Google utiliza leilões: \url{https://support.google.com/google-ads/answer/142918}.} e a venda de bens em geral. O leilão é um método universal de venda, pois qual quer coisa possível de ser vendida pode ser vendida num leilão \citet{krishna}.

Duas propriedades importantes dos leilões são o fato extraírem informação dos agentes participantes por meio dos lances e a identidade desses agentes ser irrelevante para o resultado \citet{krishna}.

Os leilões entregam um arcabouço teórico geral para entender o problema da alocação de recursos entre agentes auto-interessados, tendo aplicações muito além das vendas como a alocação de recursos computacionais em sistemas compartilhados \citet{Shoham2008}.

\section{Leilões de um único objeto}

Trataremos aqui dos formatos de leilões mais conhecidos na literatura.

\subsection{Leilão inglês}

Provavelmente o formato de leilão mais conhecido é o leilão aberto ascendente ou também conhecido por leilão inglês. É o modelo onde há um leiloeiro e diversos compradores que dão lances cada vez maiores, cobrindo o lance anterior, até que ninguém queira cobrir o último lance. Há um tempo para que os interessados possam cobrir o lance anterior. O lance de maior valor, e portanto o último, determinará quem é o vencedor.

Uma variação do leilão inglês é o leilão japonês, onde o leiloeiro anuncia os preços e, para cada preço, os compradores decidem se participam até que somente um anuncie a participação para um dado preço \citet{Shoham2008}.

Com alguma reflexão podemos ver que o lance que vence o leilão precisa ser minimamente maior ou igual ao lance do segundo colocado.

\subsection{Leilão holandês}

Ao contrário do leilão inglês, os preços, no holandês, são diminuídos com o tempo. O leiloeiro começa anunciando um preço presumidamente alto e segue diminuindo até que alguém aceite participar. O leilão é chamado de holandês, pois é o procedimento utilizado no mercado de flores de Amsterdã e quem diminui o preço é um relógio que é parado quando alguém decide comprar as flores pelo preço exibido \citet{Shoham2008}.

\subsection{Leilão de envelope fechado}

O leilão de envelope fechado funciona exatamente como o nome diz: todos os participantes submetem envelopes com seus lances que são abertos ``ao mesmo tempo'' de tal sorte que nenhum participante observa o lance do outro até a abertura dos envelopes. Como há apenas um lance por participante, o tempo despendido para comunicação é bem menor \citet{karlin2017game}. A própria entrega de lances não precisa ser realizada de forma síncrona \citet{krishna}.

Os dois modelos de leilão de envelope fechado mais conhecidos são o de primeiro (ou maior) preço e o de segundo preço, onde o valor pago pelo participante vencedor corresponde ao lance do segundo colocado.

O leilão holandês e o de primeiro são estrategicamente equivalentes, isto é, possuem a mesma forma normal \citet{krishna}. Vale ressaltar que em ambos os formatos \emph{nenhuma} informação é dada até o momento em que há a determinação do vencedor.

Já os leilões inglês e de segundo preço são \emph{parecidos}, mas não equivalentes. Sob a hipótese de independência entre os valores privados, a estratégia ótima é a mesma \citet{krishna}. Contudo, pode haver alguma diferença considerando, por exemplo, o tamanho do passo no leilão inglês \citet{Shoham2008}.

\subsection{Valoração}

Se o leiloeiro soubesse exatamente quem atribui o maior valor ao produto que está a venda, não haveria sentido em se realizar um leilão, pois bastaria o leiloeiro vender o produto diretamente. Essa incerteza em relação à valoração é uma característica inerente ao leilão \citet{krishna}.

Dependendo de como é a configuração dos lances, os demais participantes passam a conhecer, mesmo que parcialmente, a valoração de seus adversários. Num leilão inglês, à medida em que os lances são dados, mais informação sobre a valoração fica disponível para todos os participantes. Por outro lado, num leilão de envelope fechado a valoração só é conhecida quando da abertura dos envelopes.

Esse acesso à informação pode mudar o comportamento dos agentes, pois pode revelar alguma informação sobre o bem à venda que não é conhecida para todos os participantes. O impacto depende da relação entre os agentes e o bem. É razoável que se o bem for para consumo, conhecer a valoração de outros agentes não é tão relevante. Entretanto, se o objetivo for revender, vale a pena conhecer a valoração dos outros participantes para se ter uma ideia do mercado. Essa informação também é importante na ocasião de o próprio agente ter uma mera crença sobre o valor do bem que está em disputa. À medida em que mais lances são conhecidos essa crença pode ser atualizada.

O caso onde a valoração dos demais participantes é irrelevante é chamado de \textbf{valores privados independentes}. Quando todos têm uma mesma valoração a situação é chamada de \textbf{valor comum} \citet{krishna}. Quando há alguma influência, chamados de \textbf{valores privados interdependentes}, em particular, esses valores podem ser \textbf{afiliados}. De forma, imprecisa, o leilão apresenta valores afiliados se a valoração alta de um jogador torna provável que outros jogadores também tenham uma valoração alta \citet{Klemperer2004-qd}.

Quando os valores são afiliados, não vale mais a equivalência entre o leilão inglês e o leilão de segundo preço. Em, particular, as receitas dos quatro leilões podem ser ranqueadas da seguinte forma \citet{krishna}:
\begin{equation*}
	E[\text{inglês}] >= E[\text{segundo preço}] >= E[\text{primeiro preço}] = E[\text{holandês}]
\end{equation*}

Além disso, vale o princípio da ligação\footnote{\emph{Linkage principle.}}, que diz que, se o leiloeiro dispuser de alguma informação afiliada à valoração dos agentes, vale a pena revelar tal informação, mesmo se tal informação diminuir o preço. Publicizá-la revela informação sobre a valoração do agente vencedor, diminuindo seu ganho justamente por esconder informação sobre a valoração que possui, isto é, aumenta a receita esperada \citet{Shoham2008}.

O princípio da ligação possui uma aplicação prática quando do desenho de editais pelo setor público, pois em licitações de grande porte, há uma extensa gama de informações. Isso sugere que o próprio poder público, por questão de centralização, possa dispensar recursos para conseguir e liberar tais informações \citet{pellegrini2018:MSc}.

\section{O modelo de um leilão}

Dados $N$ jogadores, cada jogador $i$ atribui um valor privado $x_i$ ao objeto que está à venda e, portanto, não pagará mais do que $x_i$ por ele. Se o preço que venceu o leilão for $p$, o jogador que vencer ganhará $x_i - p$. Os outros jogadores ganharão zero. Não há restrição orçamentária sobre os agentes.

Primeiramente, vamos estudar como os jogadores determinam a função de lance, $\beta_i$. A função de lance depende exclusivamente do valor privado do jogador.

\subsection{O lance no leilão segundo preço}

\begin{proposicao}
    \label{prop:estrategia-dominante}
    Num leilão de segundo preço, o jogador $i$ não tem incentivo para dar um lance diferente de sua valoração $x_i$. Isto é, $\beta_{i}(x_i) = x_i$.
\end{proposicao}

Em um leilão inglês, de uma forma parecida, só faz sentido o jogador $i$ permanecer dando lances enquanto o preço, $p$, for menor que $x_i$, pois isso maximiza sua utilidade independente do que fizerem os outros jogadores \citet{karlin2017game}. A estratégia da proposição \ref{prop:estrategia-dominante} também majora a utilidade do jogador $i$ independente dos outros jogadores. Em ambos os casos, a estratégia é uma estratégia dominante. É importante ressaltar que essa escolha acontece \emph{independentemente do perfil de risco} dos jogadores \citet{Shoham2008}.

Contudo, se a valoração contiver um componente de valor comum, a estratégia de equilíbrio deixa de ser dar o lance igual à própria valoração ($\beta(x_i) = x_i)$. A ideia é que o valor é comum a todos os jogadores e, portanto, a valoração dos outros jogadores será levada em consideração. Como o lance vencedor na situação hipotética de $\beta(x_i) = x_i$ será a avaliação mais otimista possível sobre o valor efetivo do bem adquirido, o jogador vencedor diminuirá sua própria expectativa sobre o valor, levando-o à conclusão de que pagou demais \citet{Shoham2008}. Esse fenômeno é chamado de maldição do vencedor.

\subsection{O lance no leilão de primeiro preço}

Como os leilões holandês e de primeiro preço são estrategicamente equivalentes, basta tratarmos o leilão de primeiro preço. Mas, ao contrário do leilão de segundo preço, não há uma estratégia dominante. Teremos de adicionar algumas propriedades ao modelo.

O jogador $i$ ainda conhece sua valoração $x_i$, mas, agora, também sabe que a valoração de cada um dos jogadores segue uma distribuição de probabilidade acumulada $F$ estritamente crescente, com derivada contínua e que assume valores não nulos em algum intervalo $[0, 1]$. Essa configuração confere o caráter bayesiano ao jogo, pois plugamos o conjunto de tipos e a distribuição de probabilidade à priori ao modelo. Em particular, o sinal é a própria valoração do agente. Além disso, vamos assumir que os jogadores são \textbf{neutros ao risco}.

Recapitulando, valem as propriedades \citet{krishna}:
\begin{itemize}
	\item Independência dos valores privados;
	\item Ausência de restrição orçamentária;
	\item Simetria entre jogadores;
	\item Neutralidade a risco.
\end{itemize}

Nosso objetivo é encontrar funções $\beta_i : [0, 1] \xrightarrow{} \mathbb{R}$ que representem em equilíbrio os lances dos jogadores $i$ definidas suas valorações. Desejamos encontrar o equilíbrio de Nash simétrico, isto é, todos os jogadores usarão a mesma função de lance, $\beta_i = \beta, \forall i$ \citet{krishna}.

Primeiramente, $\beta(0) = 0$, afinal nenhum jogador tem interesse em pagar por algo que vale 0 para si. Além disso, parece razoável supor que $\beta$ é crescente, pois, quanto mais valorizado o objeto, maior tende a ser o lance.

Seja $b$ o lance do jogador $i$. Se o jogador $i$ vencer, $b > max_{j \ne i} \beta(X_j)$. Seja $Y$ a variável aleatória associada à maior valoração dentre os jogadores restantes, então:
\begin{equation}
b > \beta(Y) \Rightarrow Y < \beta^{-1}(b)    
\end{equation}
Como todas $X_i$ são independentes, a distribuição de $Y$ é a estatística de ordem do máximo entre os $n-1$ outros jogadores, isto é, $G(y) = F(x) \ldots F(x) = F(x)^{n-1}$.

O \emph{payoff} esperado do jogador $i$ é:
\begin{equation}
    P[vencer] (x_i - b) \Rightarrow \
    G(\beta^{-1}(b)) (x_i - b)
\end{equation}

Basta encontrar o $b$ que maximiza o \emph{payoff}.

\begin{proposicao}
    \label{prop:nash-primeiro-preco}
    A estratégia de equilíbrio simétrico num leilão de primeiro preço é dada por:
    \begin{equation}
        \beta(x) = \frac{1}{G(x)} \int_{0}^{x} yg(y)dy = E\left[ Y \mid Y<x \right] = x - \int_{0}^{x} \frac{G(y)}{G(x)}dy
    \end{equation}
\end{proposicao}

Vale ressaltar que o lance no leilão de primeiro preço é a valoração do jogador subtraída de um valor que é sempre positivo.

Findos os lances, agora vamos determinar quanto o vencedor pagará em cada situação.

\subsection{Preço de reserva}

Até o presente momento, o leiloeiro não dispunha de jogadas. Basicamente os jogadores davam seus lances e o bem leiloado era alocado para o jogador de maior valor privado à revelia do leiloeiro. Isso significa que potencialmente o bem pode ser vendido por $\epsilon$, o menor lance possível. Entretanto, podemos assumir que o leiloeiro também tem uma valoração para o bem à venda de tal sorte que a venda por um valor inferior não é desejável.

Adicionaremos ao jogo a possibilidade de o leiloeiro escolher um valor de reserva, $r$, que limita inferiormente o menor lance possível dado em um leilão.

Num leilão de segundo preço, o preço de reserva não muda a estratégia dos jogadores: ainda é ótimo dar o lance $\beta(x) = x$ \citet{krishna}.

Já no de primeiro preço os jogadores de valoração inferior a $r$, darão um lance de $0$ e os jogadores de valoração $r$ só vencerão na situação de todos os outros jogadores terem valoração inferior a $r$. Nesse caso $\beta(r) = r$. O restante dos jogadores dará lances de forma similar à situação sem preço de reserva \citet{krishna}.

\begin{proposicao}
	A estratégia de equilíbrio simétrico num leilão de primeiro preço com preço de reserva $r$ é dada por:
	\begin{equation}
		\beta(x) = E[max\{r, Y\} \mid Y < x] = r\frac{G(r)}{G(x)} + \frac{1}{G(x)} \int_r^x yg(y)dy
	\end{equation}
\end{proposicao}
\begin{proof}
	Prova análoga à da proposição \ref{prop:nash-primeiro-preco}.
\end{proof}

\subsection{A receita}

No leilão de primeiro preço é imediato, o pagamento esperado é dado pelo maior lance esperado, que é: $m(x) = G(x)E\left[ Y \mid Y<x \right]$, onde $G(x)$ é a probabilidade de o jogador ganhar o certame com o lance $x$.

O pagamento no leilão de segundo preço é dado pelo segundo maior lance. Como $\beta(x) = x$, o pagamento esperado no leilão de segundo preço será dado pelo segundo maior valor, resultando em: $m(x) = G(x)E\left[ Y \mid Y<x \right]$.

O pagamento \emph{ex ante} é dado por:
\begin{equation*}
	E[m(X)] = \int_0^1 y(1-F(y))g(y)dy
\end{equation*}

Como os $n$ jogadores são simétricos, a receita do leiloeiro é dada simplesmente pela multiplicação do número de jogadores pelo pagamento \emph{ex ante}:
\begin{equation*}
	E[R] = nE[m(X)] = E[Y_2]
\end{equation*}

Os detalhes aritméticos para atingir as igualdades acima estão em \citet{krishna}.

Analogamente, para o caso de preço de reserva:
\begin{equation*}
	E[m(X,r)] = r(1-F(r))G(r) + \int_r^1 y(1-F(y))g(y)dy
\end{equation*}

Seja $x_0$ a valoração do leiloeiro. É razoável supor que $r \geq x_0$. Além disso, o pagamento esperado na ocasião de todos os $n$ participantes darem lances inferiores a $r$ é justamente $F(r)^n x_0$.

A receita esperada, portanto, é dada por:
\begin{equation*}
	E[R] = n E[m(X,r)] + F(r)^n x_0
\end{equation*}

Com o objetivo de analisar a receita esperada, verifiquemos o comportamento de sua derivada, que é:
\begin{equation*}
	nG(r)\left(1-\frac{f(r)}{1-F(r)}(r-x_0)\right)
\end{equation*}

Primeiramente, se $x_0 = r$, a derivada é positiva e, portanto, não pode ser um ponto crítico. Além disso, se $x_0 > r$ e alguém ganhar pagando o preço de reserva, o leiloeiro sairá perdendo. Portanto, a fim de maximizar a receita, $r$ deve ser maior que $x_0$.

No caso de $x_0 = 0$, a receita esperada terá um ponto crítico justamente em $r = 0$. Contudo, se $\frac{f(r)}{1-F(r)}$ for limitada, o ponto será um ponto de mínimo \citet{krishna}. Portanto, $r > x_0 = 0$. Isso significa que a receita esperada com preço de reserva é maior que a receita esperada sem preço de reserva, isto é, quando $r = x_0 = 0$.

Um leilão é dito \textbf{eficiente} se aloca o bem para o jogador que mais o valoriza. Contudo, o leilão com preço de reserva maximiza a receita excluindo os jogadores com valoração inferior ao preço de reserva, podendo não ser eficiente.

Uma situação que ilustra consiste em um leilão com preço de reserva $r$, leiloeiro com valoração $x_0 = 0$ e com o jogador $i$ de maior valoração tendo $x_i < r$. O preço de reserva excluirá tal jogador do certame.

Um leilão que maximiza a receita do leiloeiro é dito um leilão \textbf{ótimo}. Pelo exposto anteriormente, um leilão ótimo pode não ser eficiente.

O quadro a seguir traz um comparativo entre os dois modelos de leilão sem preço de reserva.

\begin{table}[]
	\centering
	\begin{tabular}{@{}llllll@{}}
		\toprule
		tipo      & primeiro 					     & segundo      					& 1{\textordmasculine } com reserva  & 2{\textordmasculine } com reserva \\ \midrule
		lance     & $E\left[ Y \mid Y<x \right]$     & $x$             					& $E[max\{r, Y\} \mid Y < x]$     & $x$\\
		pagamento & $G(x)E\left[ Y \mid Y<x \right]$ & = primeiro                       & $G(x)E[max\{r, Y\} \mid Y < x]$ & = 1{\textordmasculine } com reserva \\
		receita   & $E\left[ Y_2 \right]$            & $E\left[ Y_2 \right]$            & $> E\left[ Y_2 \right]$         & $> E\left[ Y_2 \right]$ \\
	\end{tabular}
	\caption{Comparativo entre os modelos de leilão.}
	\label{tab:tabela1}
\end{table}

\section{Teorema da Equivalência de Receitas}

Pode parecer curioso que o pagamento em ambos os casos seja o mesmo e, mais, que a receita esperada auferida pelo leiloeiro seja rigorosamente a mesma, mas esse resultado é um dos teoremas mais importantes para a Teoria dos Leilões.

\begin{teorema}[Teorema da Equivalência de Receitas]
	\label{teorema:equivalencia_de_receitas}
	Considere um leilão com as restrições que utilizamos para o leilão de primeiro preço, a saber:
	\begin{itemize}
		\item $n$ jogadores cuja valoração segue uma mesma distribuição acumulada $F()$;
		\item os valores privados são independentes;
		\item $F()$ é estritamente crescente e com suporte em $[0,1]$;
		\item os jogadores são neutros a risco.
	\end{itemize}
	Se valem adicionalmente:
	\begin{enumerate}
		\item $\beta(0) = 0$;
		\item o bem é alocado para o jogador de maior valoração.
	\end{enumerate}
	A receita do leilão será a mesma de um leilão de primeiro (ou segundo) preço e, em particular, será dada exclusivamente como função do maior valor privado dentre os jogadores.
\end{teorema}

Trata-se de um teorema bastante poderoso capaz de dar tratamento para leilões cuja análise do equilíbrio não é trivial como, por exemplo, leilões onde os jogadores não conhecem o número total de jogadores, leilões onde todos pagam, leilões de $n$-ésimo preço e provas de resistência.

Também permite dar um tratamento interpretativo em termos de teoria dos jogos para fenômenos que, num primeiro momento, não se assemelham aos leilões como filas, concursos e até sistemas de litigação \citet{Klemperer2004-qd}.

O exemplo dado por \citet{Klemperer2004-qd} interpreta uma sugestão do então vice presidente dos EUA, Dan Quayle, que sugeriu reformar o sistema de litigação dos EUA da seguinte forma: a parte que perder o processo terá de pagar à parte vencedora um adicional igual ao seu próprio dispêndio com o processo. A intuição era de que aumentar os custos processuais desincentivaria as litigações.

Mas imaginemos que cada parte tenha uma valoração privada de vencer o processo em relação a perder obtidos de forma independente a partir de uma mesma distribuição de probabilidade acumulada com suporte finito. Cada parte decide de forma independente o quanto vai despender no processo de tal sorte que a parte que despender mais, ganha. Vale ressaltar que se for despendido 0, a parte perdedora pagará \emph{duas vezes zero}. Logo vale o Teorema da Equivalência de Receitas e a alocação, isto é, quem processará e quem não processará, não mudará.

O Teorema da Equivalência de receitas, num certo sentido, é dual em relação à área de estudo da Economia. Enquanto a economia quer alocar recursos escassos a partir do sistema de preços, o Teorema da Equivalência de Receitas encontra os preço - os lances - a partir de uma dada alocação.

Há um porém em relação ao Teorema da Equivalência de Receitas: embora seja verdade que os leilões que satisfizerem as condições do teorema terão a mesma receita esperada, não é verdade que todas as estratégias que levam àquela receita são necessariamente estratégias de equilíbrio \citet{Shoham2008}. Isso significa que é necessário provar que a estratégia encontrada é efetivamente uma estratégia de equilíbrio.

\section{Leilões multi-objeto}

A principal diferença entre um leilão multi-objeto para um leilão com um único objeto repousa sobre como a valoração é definida. Como agora, em vez de um único objeto à venda, existe um determinado conjunto de objetos, $K$, a valoração, $x_i$, será uma função sobre cada um dos subconjuntos de $K$, isto é, o conjunto das partes de $K$.

Um subconjunto de $K$ é chamado de pacote\footnote{\emph{Package}.}. Em particular, a valoração sobre o conjunto com nenhum objeto será zero, $x_i(\emptyset) = 0$. Sejam $S_1 \subset K$ e $S_2 \subset S_1 \subset K$, então $x_i(S_2) \leq x_i(S_1)$.

O resultado de um leilão é sempre uma alocação. No contexto de um leilão multi-objeto, a alocação entrega uma partição de $K$, isto é, uma coleção de subconjuntos disjuntos de $K$ onde esses subconjuntos estão associados aos jogadores que deram os melhores lances.

O primeiro problema é que pode haver sinergia entre os objetos. Se dois objetos $a$ e $b$ forem complementares:
\begin{equation*}
	x_i(\{a\}) + x_i(\{b\}) \leq x_i(\{a,b\})
\end{equation*}

Por outro lado, se forem substitutos:
\begin{equation*}
	x_i(\{a\}) + x_i(\{b\}) \geq x_i(\{a,b\})
\end{equation*}

Diante dessas sinergias, outro problema é que pode ser interessante para o leiloeiro dividir esses objetos em lotes. Em particular, na situação de existirem apenas dois jogadores, é sempre melhor montar um único lote \citet{krishna}.

Há um problema prático grave de natureza computacional ao se trabalhar com leilões multi-objeto: encontrar uma alocação eficiente é o mesmo que resolver um problema de programação inteira e, nesse caso, com uma instância da ordem de $2^|K|$. Sua solução exata é intratável de um ponto de vista computacional \citet{Nisan2007}.

\section{Pregões}

Qual a relação entre um leilão e um pregão? Tal pergunta foi formalizada em \citet{deCastro2010} com a prova de que a relação entre um pregão e um leilão é de dualidade, isto é, há um isomorfismo entre as estratégias empregadas num leilão (problema primal) e num pregão (problema dual). Permite-se, então, o emprego de resultados já consolidados na Teoria dos Leilões para atacar os problemas concernentes ao processo licitatório semelhantes aos da modalidade pregão. Pregões são simplesmente leilões onde se troca a palavra vendedor por comprador e se coloca o sinal de menos em todos os números que indicam preços ou lances \citet{Shoham2008}.

Um pregão de primeiro preço com valores privados independentes, jogadores neutros a risco é um jogo bayesiano onde os jogadores, chamados licitantes, obtém a partir de uma mesma e conhecida distribuição de probabilidade um sinal que representa seu \textbf{custo}. O objetivo dos licitantes é maximizar o \emph{payoff}, que agora é o preço pago, $p$, subtraído de seu custo privado, $c_i$ \citet{Bugarin2022}.

Um pregão de primeiro preço preserva as mesmas propriedades de um leilão de primeiro preço, em especial, os resultados sobre os equilíbrios e preço de reserva \citet{Bugarin2022}. Um pregão de segundo preço também preserva as propriedades de um leilão de segundo preço, em particular, a existência da estratégia fracamente dominante.

\begin{proposicao}
	\label{prop:nash-pregao-primeiro-preco}
	A estratégia de equilíbrio simétrico num pregão de primeiro preço é dada por:
	\begin{equation}
		\beta(c) = \frac{1}{1-G(c)} \int_c^1 yg(y)dy = E[Y \mid Y > c] = c + \int_c^1 \frac{1-G(y)}{1-G(c)}dy
	\end{equation}
\end{proposicao}
\begin{proof}
	Em \citet{Bugarin2022}.
\end{proof}

A proposição \ref{prop:nash-pregao-primeiro-preco} é ``idêntica'' à proposição \ref{prop:nash-primeiro-preco}.

Entretanto, nem tudo que existe no mundo dual é facilmente recuperado no mundo primal. Há considerações sobre os agentes que parecem fugir de uma interpretação trivial como a questão da qualidade dos bens ofertados em um pregão. Caso a qualidade, $q$, for meramente obtida por meio de um sinal de uma mesma distribuição de probabilidade conhecida por todos os participantes, haverá o fenômeno da seleção adversa \citet{Che1993}. Atentamos que, por lei, só é possível realizar pregões na modalidade menor preço ou maior desconto. 

A qualidade também pode versar sobre caraterística não-contratável, isto é, que não pode figurar diretamente no edital do certame. Há a possibilidade de se empregar um mecanismo onde há a previsão não só de um preço de reserva, mas de um preço mínimo para a aquisição dos bens, o \emph{LoLA - Lowball Lottery Auction} \citet{villa:2022}.

Quando o problema da qualidade se torna endógeno, há a possibilidade de se empregar uma função de \emph{score}, $s(q)$, cujo objetivo é determinar a qualidade por meio de um parâmetro de eficiência, este, sim, obtido a partir de uma distribuição de probabilidade conhecida por todos os participantes. Isso dá substância ao lema:

\begin{lema}
	Com leilões (reversos) de primeiro e de segundo preço, a qualidade é escolhida em $q_s(\theta)$ para $\theta \in [0, 1]$ onde:
	\begin{equation}
		q_s(\theta) = \text{argmax } s(q) - c(q, \theta)
	\end{equation}
\end{lema}
\begin{proof}
	Em \citet{Che1993}.
\end{proof}

Quando não é possível utilizar alguma estratégia de \emph{score}, há outros mecanismos possíveis.

Uma possibilidade é empregar um mecanismo de ofertas sequenciais onde o comprador, no nosso caso o órgão público, decide se participa ou não. Se participar, dado um conjunto de preços $(k_i)$, é realizada uma oferta para o vendedor $i$ que pode aceitar ou recusar. Nesse caso a oferta é realizada para o próximo vendedor. O jogo acaba quando algum vendedor aceita a oferta ou quando se esgota o número de vendedores. Esse jogo tem um único equilíbrio perfeito de subjogo e o comprador decide se participa justamente se seu \emph{payoff} esperado for estritamente positivo \citet{Manelli1995}.

Saindo da qualidade, a dinâmica de preços de reserva formado pela mediana de aquisições anteriores, possibilidade sugerida pela lei, leva o preço a convergir ao preço mínimo exequível, ampliando o risco de o contrato não se concretizar. Essa dinâmica pode levar ao problema da maldição do vencedor \citet{Signor2022}.

O mecanismo de registro de preços prevê a possibilidade de diferentes órgãos aderirem. Dependendo do custo para viabilizar a ata de registro de preço, o órgão público pode optar por não realizar o certame dessa forma, pois uma quantidade excessiva de caronas pode tornar o \emph{payoff} muito baixo se comparado ao custo de sua realização \citet{barbosa2013}.