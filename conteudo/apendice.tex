\chapter{Apêndice}
\label{cap:apendice}

\section{Estratégia dominante no leilão de segundo preço}
\begin{lema}
	Num leilão de segundo preço, o jogador $i$ não tem incentivo para dar um lance diferente de sua valoração $x_i$. Isto é, $\beta_{i}(x_i) = x_i$.
\end{lema}
\begin{proof}
	Vamos olhar apenas para as situações onde o $i$ pode ganhar o leilão.
	
	Se $i$ desviar e jogar $y_i > x_i$, um jogador $j$ pode jogar $y_j : y_i > y_j > x_i$. Nesse caso, $i$ ganha o leilão, mas pagará $y_j$, ficando com $x_i - y_j < 0$. Se tivesse jogado $x_i$, perderia 0.
	
	Se $i$ desviar e jogar $w_i < x_i$, um jogador $j$ pode jogar $z_j : x_i > z_j > z_i$. Nesse caso, $i$ perde o leilão, mas poderia ter ganho $x_i - z_j > 0$.
\end{proof}

\section{Lance no leilão de primeiro preço}
\begin{lema}
	A estratégia de equilíbrio simétrico num leilão de primeiro preço é dada por:
	\begin{equation}
		\beta(x) = \frac{1}{G(x)} \int_{0}^{x} yg(y)dy = E\left[ Y \mid Y<x \right] = x - \int_{0}^{x} \frac{G(y)}{G(x)}dy
	\end{equation}
\end{lema}
\begin{proof}
	Como todos os jogadores implementam a melhor resposta em um equilíbrio de Nash, vamos maximizar para $b$. Por simplicidade, vamos remover o $i$ subscrito.
	\begin{align*}
		\frac{d G(\beta^{-1}(b)) (x - b)}{db} & = 0 & \Rightarrow \\
		\frac{ g( \beta^{-1}(b) ) }{ \beta'( \beta^{-1}(b) ) }(x - b) - G( \beta^{-1}(b) ) & = 0 & \Rightarrow \\
		G( \beta^{-1}(b) )\beta'( {\beta^{-1}(b)} ) & = g( \beta^{-1}(b) )(x - b) & \Rightarrow \\
		g(\beta^{-1}(b))b + G(\beta^{-1}(b))\beta'({\beta^{-1}(b)}) & = g(\beta^{-1}(b))x
	\end{align*}
	
	Aplicando a hipótese de simetria, $b = \beta(x)$, segue que:
	\begin{align*}
		g(x)\beta(x) + G(x)\beta'(x) & = xg(x) & \Rightarrow \\
		\frac{d}{dx} \beta(x)G(x) & = xg(x) & \Rightarrow \\
		\left[ \beta(x)G(x) \right]_{0}^{x} & = \int_{0}^{x} yg(y)dy & \Rightarrow \\
		\beta(x)G(x) - \cancelto{0}{\beta(0)}G(0) & = \int_{0}^{x} yg(y)dy
	\end{align*}
\end{proof}