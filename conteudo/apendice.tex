\chapter{Demonstrações}
\label{cap:apendice}

\section{Estratégia dominante no leilão de segundo preço}
\begin{proof}[Prova do Lema \ref{lema:estrategia-dominante}]
	Vamos olhar apenas para as situações onde o $i$ pode ganhar o leilão.
	
	Se $i$ desviar e jogar $y_i > x_i$, um jogador $j$ pode jogar $y_j : y_i > y_j > x_i$. Nesse caso, $i$ ganha o leilão, mas pagará $y_j$, ficando com $x_i - y_j < 0$. Se tivesse jogado $x_i$, perderia 0.
	
	Se $i$ desviar e jogar $z_i < x_i$, um jogador $j$ pode jogar $z_j : x_i > z_j > z_i$. Nesse caso, $i$ perde o leilão, mas poderia ter ganho $x_i - z_j > 0$.
	
	Portanto $\beta(x_i) = x_i$.
\end{proof}

\section{Lance no leilão de primeiro preço}
\begin{proof}[Prova do Lema \ref{lema:nash-primeiro-preco}]
	Como todos os jogadores implementam a melhor resposta em um equilíbrio de Nash, vamos maximizar para $b$. Por simplicidade, vamos remover o $i$ subscrito.
	\begin{align*}
		\frac{d G(\beta^{-1}(b)) (x - b)}{db} & = 0 & \Rightarrow \\
		\frac{ g( \beta^{-1}(b) ) }{ \beta'( \beta^{-1}(b) ) }(x - b) - G( \beta^{-1}(b) ) & = 0 & \Rightarrow \\
		G( \beta^{-1}(b) )\beta'( {\beta^{-1}(b)} ) & = g( \beta^{-1}(b) )(x - b) & \Rightarrow \\
		g(\beta^{-1}(b))b + G(\beta^{-1}(b))\beta'({\beta^{-1}(b)}) & = g(\beta^{-1}(b))x
	\end{align*}
	
	Aplicando a hipótese de simetria, $b = \beta(x)$, segue que:
	\begin{align*}
		g(x)\beta(x) + G(x)\beta'(x) & = xg(x) & \Rightarrow \\
		\frac{d}{dx} \beta(x)G(x) & = xg(x) & \Rightarrow \\
		\left[ \beta(x)G(x) \right]_{0}^{x} & = \int_{0}^{x} yg(y)dy & \Rightarrow \\
		\beta(x)G(x) - \cancelto{0}{\beta(0)G(0)} & = \int_{0}^{x} yg(y)dy & \Rightarrow \\
		\beta(x) & = \frac{1}{G(x)} \int_{0}^{x} yg(y)dy
	\end{align*}
\end{proof}

\section{Lance no leilão de primeiro preço com opção de aquisição}
\begin{proof}[Prova do Lema \ref{lema:nash-primeiro-preco-opcao}]
	Como todos os jogadores implementam a melhor resposta em um equilíbrio de Nash, vamos maximizar para $b$.
	
	\begin{align*}
		& \frac{d H(\beta^{-1}(b)) G(\beta^{-1}(b)) (x - b)}{db} = 0 & \Rightarrow \\
		\frac { h(\bbeta}{\dinvbeta} \left[ G(\bbeta)(x - b) \right] & + H(\bbeta) \left[ \frac{g(\bbeta)}{\dinvbeta}(x - b) - G(\bbeta) \right] = 0 & \Rightarrow \\
		h(\bbeta) \left[ G(\bbeta)(x - b) \right] & + H(\bbeta) \left[ g(\bbeta)(x - b) \right] = H(\bbeta)G(\bbeta)\dinvbeta
	\end{align*}

	Aplicando a hipótese de simetria, $b = \beta(x)$, segue que:
	\begin{align*}
		h(x)G(x)(x-\beta(x)) + H(x)g(x)(x-\beta(x)) & = H(x)G(x)\beta'(x) & \Rightarrow \\
		\left[ h(x)G(x) + H(x)g(x) \right]x & = \left[ h(x)G(x) + H(x)g(x) \right]\beta(x) + H(x)G(x)\beta'(x) & \Rightarrow \\
		\frac {d H(x)G(x) }{dx} x & = \frac{d H(x)G(x)\beta(x)}{dx} & \Rightarrow \\
		\int_0^x \frac {d H(y)G(y) }{dy} ydy & = \int_0^x \frac{d H(y)G(y)\beta(y)}{dy}dy & \Rightarrow \\
    	\int_0^x (H(y)G(y))'ydy & = H(x)G(x)\beta(x) - \cancelto{0}{H(0)G(0)\beta(0)} & \Rightarrow \\
    	\left[ H(y)G(y)y \right]_0^x - \int_0^x H(y)G(y)dy & = H(x)G(x)\beta(x) & \Rightarrow \\
    	H(x)G(x)x - \int_0^x H(y)G(y)dy & = H(x)G(x)\beta(x) & \Rightarrow \\
    	\beta(x) & = x - \int_0^x \frac{H(y)G(y)}{H(x)G(x)}dy
	\end{align*}

\end{proof}