\unnumberedchapter{Introdução}
\label{cap:introducao}

\enlargethispage{.5\baselineskip}

As compras no setor público são, em geral, disciplinadas pelo Artigo 37 da Constituição Federal, que diz, em seu inciso XXI, que \emph{as obras, serviços, compras e alienações serão contratados mediante processo de licitação pública}.

Com o objetivo de realizar suas atividades a contento, os agentes públicos precisam adquirir bens e serviços, atendendo aos princípios que norteiam a administração pública\footnote{Princípio da legalidade, moralidade, eficiência, economicidade\ldots Podemos verificar todos os princípios no Artigo 5º da Lei Nº 14133.}. Contudo, o processo licitatório é bastante diferente da compra que o consumidor realiza usualmente.

O primeiro entrave é que, embora seja possível sintetizar as características de um bem utilizando marca e modelo, licitar por marca e modelo não é permitido sem as devidas justificativas. Segue que é necessário o trabalho de especificar o bem a partir de suas características relevantes, levando-nos ao segundo entrave: nem toda característica do bem está disponível para montar a especificação. Ademais, a própria marca e modelo podem ser \emph{proxies} para a qualidade do produto desejado.

O problema da especificação não é o único conjunto de entraves. Outra dificuldade do processo licitatório reside justamente em sua organização propriamente dita. Em particular, desejamos garantir ampla participação. Os principais instrumentos que temos à disposição são as condições presentes no edital, a organização dos lotes e, preservadas as hipóteses da Lei de Licitações, a escolha da modalidade. A Lei de Licitações também prevê o uso de um valor estimado\footnote{Artigos 23 e 58. Chamamos de preço de reserva.}, não permitindo, por exemplo, a supressão de tal valor no certame mesmo que seja teoricamente interessante.

A organização dos lotes consiste em decidir qual é a maneira ótima de dividir os diversos bens que se deseja adquirir num processo licitatório. Por exemplo, desejando-se adquirir os bens $A$ e $B$, é melhor proceder de forma separada? Ou seria mais adequado montar um lote com ambos os bens? A própria Lei de Licitações, em seu Artigo 40, sugere considerar a viabilidade da divisão do objeto em lotes para aproveitar as peculiaridades do mercado local e ampliar a competição.

A modalidade relevante para o presente trabalho é o pregão. O sistema de registro de preços é considerando um procedimento auxiliar. No pregão, o agente público escolhe tanto os bens quanto a quantidade demandada desses bens no momento da elaboração da etapa de planejamento da contratação. Homologado o processo, a empresa vencedora entrega a cesta de produtos pelo menor preço.

No registro de preços, por sua vez, é formado um contrato entre a entidade pública e a empresa vencedora para que seja ofertada \textbf{até} uma dada quantidade de bens, pelo prazo de um ano, pelo valor vencedor ofertado no certame. Isso significa que a entidade pública pode demandar, inclusive, zero unidades. O Artigo 40 da Lei de Licitações sugere o \emph{processamento por meio de sistema de registro de preços, quando pertinente}. Uma vantagem para o fornecedor do uso do sistema de registro de preços é que, em geral, é vedada a participação do órgão público em dois registros de preço com os mesmos objetos\footnote{A exceção é se o quantitativo registrado for inferior ao máximo previsto no edital.}. Do lado do órgão público, o registro de preços traz algumas vantagens como previsibilidade e celeridade, afinal não será necessário outro processo licitatório, que usualmente dura meses desde a formação da demanda até o objeto ser entregue. A aquisição por registro de preços tende a evitar problemas com estoque e a aumentar a precisão do dispêndio público. A Tabela \ref{tab:tabela1} exemplifica uma situação onde a licitação ocorreu no período 1 com a demanda total prevista de 60 unidades do objeto, tendo sido demandadas 50 unidades, ou seja, o uso do objeto diminuiu em relação ao estimado inicialmente.

\begin{table}[]
	\label{tab:tabela-rp}
	\centering
	\begin{tabular}{@{}llll@{}}
		\toprule
		período & regular & registro & evento                 \\ \midrule
		1       & 60      & 15       &                        \\
		2       & 0       & 15       &                        \\
		3       & 0       & 10       & uso do objeto diminuiu \\
		4       & 0       & 10       &                        \\
		total   & 60      & 50       &                        \\ \bottomrule
	\end{tabular}
	\caption{Exemplo comparando a aquisição de um dado objeto nas duas modalidades. No período 3, o uso do objeto diminuiu, diminuindo a quantidade efetivamente demandada.}
\end{table}

A escolha do formato do processo licitatório, por sua vez, pode tanto espantar fornecedores, eventualmente fracassando, quanto fomentar a competição, garantindo resultados potencialmente desejáveis. O fornecedor, no exemplo dado na Tabela \ref{tab:tabela-rp}, vendeu 10 unidades a menos na modalidade de registro de preços, mas, por outro lado, pode ter suavizado suas vendas ao longo dos períodos.

A pergunta que motiva o presente trabalho se deriva dessa possível ambiguidade. Como funciona o registro de preços enquanto mecanismo? Quais problemas ele resolve? Quais problemas ele causa?

No capítulo \ref{cap:revisao} há uma revisão bibliográfica sobre a teoria de leilões e sua interface com os processos licitatórios. O capítulo \ref{cap:14133} versa sobre o procedimento da logística pública segundo a Lei n{\textordmasculine } 14.133. O capítulo \ref{cap:modelo} trata de alguns modelos associados ao sistema de registro de preços. Por fim, o capítulo \ref{cap:conclusoes} traz algumas reflexões à luz da literatura e dos modelos trazidos no presente trabalho.